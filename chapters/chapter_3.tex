
\newpage
\begin{center}
\textbf{\large ГЛАВА 3 \\ Диффузия на жидких бинодальях: влияние дальнодействия силы притяжения}
\end{center}
\refstepcounter{chapter}

\addcontentsline{toc}{chapter}{ГЛАВА 3. Диффузия на жидких бинодальях: влияние дальнодействия силы притяжения}

В этой работе мы используем моделирование молекулярной динамики для расчета фазовых диаграмм обобщенных систем Леннарда-Джонса с различными показателями притяжения. Оценены коэффициенты диффузии и подвижности (обратной диффузии) и проанализированы спектры коллективного возбуждения на жидких бинодалиях. Отмечено, что зависимость коэффициента подвижности от температуры является линейной в широком диапазоне температур, а ее наклон увеличивается с увеличением показателя притяжения. В начале нелинейной зависимости подвижности от температуры дисперсионные соотношения коллективных возбуждений жидкости обнаруживают переход от колебательного к монотонному ходу.

\section{Введение}
\label{MACR-SecIntroduction}

Диффузия играет важную роль в различных процессах переноса массы, начиная от науки и техники и заканчивая живой природой.
Он играет решающую роль в биологических процессах~\cite{10.1016/j.bbagen.2013.09.037, 10.1038/s41598-018-22643-9}, а также в механизмах и кинетике химических реакций. Знание механизмов диффузии позволит добиться значительного прогресса в новых биотехнологиях и медицине, решить важные проблемы химической и фармакологической промышленности и не только~\cite{10.1002/3527602836}.

Процесс диффузии очень хорошо изучен в газах и твердых телах. Например, очень подробное знание процесса диффузии достигается в кристаллических системах~\cite{10.1016/0079-6816(95)00039-2} из-за его практической важности в металлургии для легирования~\cite{10.1016/s0924-0136(96)02826-9, 10.1016/j.actamat.2015.10.010, 10.1134/s1063783411110308} и эксплуатации полупроводниковой электроники~\cite{10.1103/physrevlett.84.4220, 10.1016/j.physrep.2009.10.003}.
