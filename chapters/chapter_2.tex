\newpage
\begin{center}
  \textbf{\large 2. ВЫБОР ТЕХНОЛОГИЙ И ПРОЕКТИРОВАНИЕ АИС}
\end{center}
\refstepcounter{chapter}
\addcontentsline{toc}{chapter}{2. ВЫБОР ТЕХНОЛОГИЙ И ПРОЕКТИРОВАНИЕ АИС}


\section{Анализ компонентов системы и требований к ним}
АИС "МТУСИ Расписание" включает следующие компоненты: 
\begin{enumerate}
  \item Мобильное приложение для просмотра расписания.
  \item ICS API для экспорта расписания в формате iCalendar.
  \item Парсер для сбора и обработки данных расписания.
  \item API для доступа к данным расписания.
  \item База данных для хранения данных расписания.
\end{enumerate}


\subsection{Мобильное приложение}
Мобильное приложение будет являться клиентом АИС "МТУСИ Расписание", 
так как именно оно будет использоваться пользователями для просмотра расписания.
Приложение должно соответствовать ряду функциональных требований:
\begin{enumerate}
  \item Отображение актуального расписания занятий для студентов и преподавателей МТУСИ.
  \item Возможность просмотра расписания на разные периоды: день, неделя, месяц.
  \item Поиск и фильтрация занятий по различным параметрам, например, по преподавателям, аудиториям или группам.
\end{enumerate}

Так же приложение должно соответствовать ряду нефункциональных требований:
\begin{enumerate}
  \item Интуитивный и простой интерфейс пользователя.
  \item Поддержка различных мобильных платформ, таких как iOS и Android.
  \item Быстрый и отзывчивый интерфейс, обеспечивающий плавное взаимодействие с приложением.
  \item Высокая скорость разработки приложения.
\end{enumerate}


\subsection{Парсер расписания}
Парсер расписания будет отвечать за сбор и обработку данных расписания. 

   Функциональные требования:
   \begin{enumerate}
      \item Автоматический сбор данных о расписании занятий из источников, таких как веб-сайт МТУСИ или электронные таблицы.
      \item Обработка и преобразование полученных данных в удобный формат для дальнейшего использования.
      \item Обновление расписания с определенной периодичностью или по запросу.
      \item Обработка изменений в расписании и обновление данных соответствующим образом.
    \item Нефункциональные требования:
      \item Надежность и стабильность парсера, чтобы обеспечить актуальность данных расписания.
      \item Масштабируемость, чтобы справиться с возрастающим объемом данных и изменениями формата расписания.
      \item Поддержка различных источников данных и форматов расписания.
      \item Эффективная обработка и хранение данных расписания.

3. GraphQL API:
   Функциональные требования:
      \item Предоставление гибкого и эффективного API для доступа к данным расписания.
      \item Возможность выполнения сложных запросов, включая фильтрацию, сортировку и агрегацию данных.
    

 Поддержка механизма аутентификации и авторизации, чтобы ограничить доступ к данным расписания.
      \item Предоставление документации API для разработчиков, объясняющей доступные запросы и структуру данных.
   Нефункциональные требования:
      \item Высокая производительность и низкая задержка при выполнении запросов.
      \item Хорошая масштабируемость, чтобы обслуживать большое количество запросов и пользователей.
      \item Безопасность данных и защита от несанкционированного доступа.
      \item Поддержка мониторинга и журналирования для отслеживания использования и производительности API.
    \end{enumerate}

В итоге, разработка АИС "МТУСИ.Расписание" требует создания удобного мобильного приложения для просмотра расписания, парсера для сбора и обработки данных расписания, а также GraphQL API для гибкого доступа к данным. При выборе технологий для реализации каждого из компонентов следует учитывать функциональные и нефункциональные требования, а также особенности разрабатываемой системы.