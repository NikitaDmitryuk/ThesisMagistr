
\begin{center}
\textbf{\large ЗАКЛЮЧЕНИЕ}
\end{center}
\refstepcounter{chapter}
\addcontentsline{toc}{chapter}{ЗАКЛЮЧЕНИЕ}


Основные результаты магистерской квалификационной работы:
\begin{enumerate}

    \item Был продемонстрирован новый подход к описанию плавления в молекулярных системах, основанный на $\lambda^2$--параметре, рассчитываемый на основе разбиения системы частиц на ячейки вороного. Показано, что разработанная модель демонстрирует существенно нелинейное поведение. Кроме того, предложенная модель позволяет с высокой степенью детализации изучать зародышеобразование в различных режимах перегрева и эволюцию жидких зародышей.

    \item Исследовано влияние формы потенциала парного взаимодействия на фазовые диаграммы и подвижность частиц в жидкой фазе. Рассчитаны кривые сосуществования газа и жидкости для потенциалов с переменной силой притяжения. Установлено, что с увеличением дальнодействия потенциала температуры тройной и критической точек, а также их отношение $T_{\rm CP}/T_{\rm TP}$ также увеличиваются. Были рассчитаны коэффициент диффузии и коэффициент подвижности на жидких бинодалиях. Обнаружено, что температурная зависимость подвижности линейна в широком диапазоне температур с тем большим наклоном, чем меньше диапазон притяжения. Установлено, что начало нелинейной температурной зависимости подвижности при высоких температурах совпадает с переходом дисперсионных зависимостей коллективных возбуждений от осциллирующего к монотонному виду.

    \item Разработан новый метод распознавания фаз, основанный на алгоритме кластеризации DBSCAN. В совокупности с алгоритмом выделения поверхности метод позволяет с высокой точность рассчитывать фазовые диаграммы систем с различной плотностью и формами кластеров. Проведен сравнительный анализ различных методов построения фазовых диаграмм. Показано, что новый метод распознавания фаз может быть применен к переохлажденным системам частиц для анализа скорости нуклеации при различном дальнодействии притяжения.

\end{enumerate}
