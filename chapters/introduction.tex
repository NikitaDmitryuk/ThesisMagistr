\newpage
\begin{center}
  \textbf{\large АННОТАЦИЯ}
\end{center}


Объектом исследования данной работы стали системы частиц, взаимодействующих посредством обобщенного потенциала Леннарда-Джонса с переменной степенью притяжения.
С помощью данных систем выявлялось влияние дальнодействия притяжения на фазовые диаграммы, роль дальнодействия в транспортных свойствах, а также влияние на спектры возбуждений.
Для решения поставленной задачи методом молекулярной динамики были смоделированы системы.
Пост обработка проводилась с использованием MATLAB и python.

В ходе работы были проведены моделирования систем с различным дальнодействием притягивающей ветви потенциала взаимодействия.
Рассчитаны фазовые диаграммы систем, их транспортные свойства и спектры возбуждений на бинодали жидкость-газ.
Предложен новый метод классификации частиц на конденсат, газ и поверхность в системах с фазовым расслоением.
На основании полученных результатов, было впервые выявлено влияние дальнодействия притяжения в молекулярных системах на фазовые диаграммы, положения тройных и критических точек и корреляции транспортных свойств со спектрами возбуждений на бинодали жидкость-газ.
С помощью нового метода классификации частиц, основанного на методе кластеризации DBSCAN (Density-Based Spatial Clustering of Applications with Noise), был разработан метод построения фазовых диаграмм и анализа нуклеации.

\onehalfspacing
\setcounter{page}{2}

\newpage
\renewcommand{\contentsname}{\centerline{\large СОДЕРЖАНИЕ}}
\tableofcontents

\newpage
\begin{center}
  \textbf{\large ВВЕДЕНИЕ}
\end{center}
\addcontentsline{toc}{chapter}{ВВЕДЕНИЕ}


\textbf{Актуальность}

За прошедшие годы в процессе диффузии в жидкостях был достигнут определённый прогресс~\cite{FrenkelBook,HansenBook,GrootBook,MarchBook}, однако знания в данной области все еще довольно ограничены.
Существуют приблизительные соотношения, которые могут с разной степенью точности описать диффузию в различных системах.
Простейшей оценкой температурной зависимости коэффициента диффузии в жидкости является закон Аррениуса~\cite{10.1126/science.278.5336.257}.
Однако пренебрежение динамической вязкостью и особенностями реального потенциала взаимодействия делает данный закон непригодным для точного определения коэффициента диффузии в широком диапазоне температур.
Среди других соотношений диффузии в жидкостях, упомянем избыточное энтропийное масштабирование коэффициентов переноса~\cite{10.1103/physreva.15.2545, 10.1038/381137a0, 10.1063/1.5055064}, соотношение температуры замерзания и плотности~\cite{10.1103/physreve.62.7524, 10.1063/1.5022058, 10.1063/1.5044703, 10.1103/physreve.103.042122}, а также соотношение Стокс-Эйнштейна между коэффициентами вязкости диффузии и сдвига~\cite{10.1063/1.446338, 10.1002/BBPC.19900940313, 10.1103/physreve.95.052122, 10.1063/1.5080662, 10.1080/00268976.2019.1643045}.
Существуют методы, позволяющие достаточно точно прогнозировать коэффициент диффузии в конкретных системах, в том числе в широком интервале температур, вплоть до критической точки и в закритической области~\cite{10.1063/1.1607953, 10.1016/j.camwa.2019.11.012, 10.1063/1.441097}.
Получены обширные результаты численного моделирования~\cite{10.1063/1.1786579, 10.1016/j.fluid.2011.03.002}.
В настоящее время применяются методы машинного обучения~\cite{10.1063/5.0011512}.
Но остаются открытыми следующие важные вопросы:
\begin{enumerate}
\item Какое влияние имеет потенциал взаимодействия между частицами на температурную зависимость коэффициента диффузии;
\item Насколько важны корреляции между спектрами возбуждений и транспортными свойствами.
\end{enumerate}

\newpage

\textbf{Цель магистерской квалификационной работы} -- установить связь \\ дальнодействия притяжения потенциала взаимодействия и спектров возбуждений с транспортными свойствами жидкостей, а также выявить влияние дальнодействия притяжения на скорость нуклеации.

\textbf{Задачи магистерской квалификационной работы:}
\begin{enumerate}
\item Расчет фазовых диаграмм для 2D и 3D систем частиц, взаимодействующих посредством обобщенного потенциала Леннарда-Джонса с различными степенями притяжения. 
\item Адаптация метода кластеризации данных DBSCAN для изучения молекулярных систем и его сравнение с другими методами.
\item Расчет и анализ транспортных свойств и коллективных возбуждений на жидкостных бинодалях.
\item Применение нового метода распознавания фаз для изучения скорости нуклеации в переохлажденных системах Леннарда-Джонса с различным дальнодействием притяжения. 
\end{enumerate}


\textbf{Научной новизной обладают следующие результаты магистерской
  квалификационной работы:}
\begin{enumerate}
\item Установлено, что подвижность имеет линейную температурную зависимость в широком диапазоне на бинодали жидкость-газ.
\item При увеличении дальнодействия потенциала увеличивается отношение температур критической к тройной точке.
  Кроме того, при этом уменьшается наклон температурной зависимости подвижности.
\item Отклонение подвижности от линейной зависимости при высоких температурах коррелирует с переходом спектров возбуждений от осцилирующего к монотонному виду.
\end{enumerate}


\textbf{Апробация} основных результатов магистерской квалификационной работы проводилась на следующих конференциях:
\begin{enumerate}
\item XX Школа-конференция молодых ученых <<Проблемы физики твердого тела и высоких давлений>>, Сочи, 16-26 сентября 2021г.
\item Современные тенденции развития функциональных материалов, Сочи, 11-14 ноября 2021г.
\item Dynamic phenomena workshop 2022.
\end{enumerate}


\textbf{Публикации:}
\begin{enumerate}
\item Kryuchkov, N. P., Dmitryuk, N. A., Li, W., Ovcharov, P. V., Han, Y., Sapelkin, A. V., and Yurchenko, S. O. (2021). \\ Mean-field model of melting in superheated crystals based on a single \\ experimentally measurable order parameter. Scientific reports, 11(1), 1-15.
\item Yakovlev, E. V., Kryuchkov, N. P., Korsakova, S. A., Dmitryuk, N. A., Ovcharov, P. V., Andronic, M. M., ... and Yurchenko, S. O. (2022). 2D colloids in rotating electric fields: A laboratory of strong tunable three-body interactions. Journal of Colloid and Interface Science, 608, 564-574.
\item Tsiok, E. N., Fomin, Y. D., Gaiduk, E. A., Tareyeva, E. E., Ryzhov, V. N., Libet, P. A., ... Yurchenko, S. O. (2022). The role of attraction in the phase diagrams and melting scenarios of generalized 2D Lennard-Jones systems. The Journal of Chemical Physics, 156(11), 114703.
\end{enumerate}
