%\chapter*{Введение}
%\addcontentsline{toc}{chapter}{Введение}

\newpage
\begin{center}
\textbf{АННОТАЦИЯ}
\end{center}
%\refstepcounter{chapter}
\addcontentsline{toc}{chapter}{АННОТАЦИЯ}

В данной работе объектом исследования были системы частиц, взаимодействующих посредством обобщенного потенциала Леннарда-Джонса, на которых выявлялась роль дальнодействия притяжения на фазовые диаграммы вещества, его роль в транспортных свойствах а также влияние на спектры возбуждений. Для решения данной задачи были использованы методы молекулярной динамики. Пост обработка проводилась с использованием MATLAB и python.

В ходе работы были проведены моделирования систем с различным дальнодействием притягивающей ветви потенциала взаимодействия.
Рассчитаны фазовые диаграммы веществ, а также их транспортные свойсва и спектры возбуждений на бинодали жидкость-газ.
Предложен новый метод классификации частиц на конденсат, газ и поверхность в системах с фазовым расслоением.
На основании полученных зависимостей, впервые выявлено влияние дальнодействия притяжения в молекулярных системах на фазовые диаграммы вещества, положение тройной и критической точки и корреляции транспортных свойств со спектрами возбуждений на бинодали жидкость-газ.
С помощью нового метода классификации частиц предложен новый способ расчета фазовых диаграмм и анализа нуклеации.



\onehalfspacing
\setcounter{page}{2}
\renewcommand{\contentsname}{\centerline{\Large{Cодержание}}}
\tableofcontents
\addtocontents{toc}{\protect\thispagestyle{fancy}}
\renewcommand{\contentsname}{\centerline{\Large{Cодержание}}}

\newpage
\begin{center}
\textbf{ВВЕДЕНИЕ}
\end{center}
%\refstepcounter{chapter}
\addcontentsline{toc}{chapter}{ВВЕДЕНИЕ}



\textbf{Актуальность}

Наше понимание процесса диффузии в жидкостях остается довольно ограниченным и фрагментарным, хотя за прошедшие годы был достигнут некоторый прогресс~\cite{FrenkelBook,HansenBook,GrootBook,MarchBook}.
Существуют приближенные скейлинги и соотношения, которые могут с разной степенью точности описывать диффузию в разных системах. Пожалуй, простейшей оценкой температурной зависимости коэффициента диффузии в жидкости является закон Аррениуса~\cite{10.1126/science.278.5336.257}. Однако пренебрежение динамической вязкостью и другими особенностями реального потенциала взаимодействия делает его непригодным для точного определения коэффициента диффузии в широком диапазоне температур. Среди других полезных соотношений, касающихся диффузии в жидкостях, можно упомянуть избыточное энтропийное масштабирование коэффициентов переноса~\cite{10.1103/physreva.15.2545, 10.1038/381137a0, 10.1063/1.5055064}, их скейлинги температуры замерзания и плотности~\cite{10.1103/physreve.62.7524, 10.1063/1.5022058, 10.1063/1.5044703, 10.1103/physreve.103.042122}, а также соотношение Стокс-Эйнштейна между коэффициентами вязкости диффузии и сдвига~\cite{10.1063/1.446338, 10.1002/BBPC.19900940313, 10.1103/physreve.95.052122, 10.1063/1.5080662, 10.1080/00268976.2019.1643045}. Существуют методы, позволяющие достаточно точно прогнозировать диффузию в конкретных системах, в том числе в широком интервале температур, вплоть до критической точки и в закритической области~\cite{10.1063/1.1607953, 10.1016/j.camwa.2019.11.012, 10.1063/1.441097}. Доступны обширные результаты численного моделирования~\cite{10.1063/1.1786579, 10.1016/j.fluid.2011.03.002}. В настоящее время применяются методы машинного обучения~\cite{10.1063/5.0011512}.
Однако остаются открытыми такие важные вопросы, как влияние потенциала взаимодействия между частицами на температурную зависимость коэффициента диффузии и насколько важны корреляции между спектрами возбуждения частиц и транспортными свойствами.

\newpage

\textbf{Цель работы} -- установить связь дальнодействия потенциала и спектров возбуждений с транспортными свойствами жидкостей, а также влияние на скорость нуклеации.

\textbf{Задачами работы являются:}
\begin{itemize}
\item Расчет фазовых диаграмм для 2D и 3D систем частиц, взаимодействующих посредством обобщенного потенциала Леннарда--Джонса с различными степенями притяжения.
\item Адаптация метода кластеризации данных DBSCAN для изучения молекулярных систем и его сравнение с другими методами.
\item Расчет и анализ транспортных свойств и коллективных возбуждений на жидкостных бинодалях.
\item Применение нового метода распознавания фаз для изучения скорости нуклеации в переохлажденных системах Леннарда-Джонса с различным дальнодействием притяжения.
\end{itemize}
