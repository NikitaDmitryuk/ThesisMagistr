%\chapter*{Введение}
%\addcontentsline{toc}{chapter}{Введение}

\newpage
\begin{center}
\textbf{АННОТАЦИЯ}
\end{center}
%\refstepcounter{chapter}
\addcontentsline{toc}{chapter}{АННОТАЦИЯ}

В данной работе объектом исследования были системы с обобщенным потенциалом взаимодействия \\ Леннарда-Джонса, на которых выявлялась роль дальнодействия притяжения на фазовые диаграммы вещества, а так же его роль в диффузии. Для решения данной задачи были использованы методы молекулярной динамики и пост-обработка с использованием MATLAB и математических и графических библиотек для языка Python.

 В ходе работы были проведены множественные моделирования систем с различными потенциалами взаимодействия, модернизированы методы пост-обработки систем с разрешением отдельных частиц с помощью разбиения на ячейки Вороного. Реализован программный пакет для определения тройных и критических точек в веществе по фазовой диаграмме, а также нахождения температуры наибольших флуктуаций плотности системы.
 
 На основании полученных зависимостей, впервые выявлено влияние дальнодействия притяжения в молекулярных системах на фазовые диаграммы вещества, положение тройной и критической точки и параметры переноса. Предложен способ расчета сжимаемости и скорости звука, используя только координаты частиц.


\onehalfspacing
\setcounter{page}{2}
\renewcommand{\contentsname}{\centerline{\Large{Cодержание}}}
\tableofcontents
\addtocontents{toc}{\protect\thispagestyle{fancy}}
\renewcommand{\contentsname}{\centerline{\Large{Cодержание}}}

\newpage
\begin{center}
\textbf{ВВЕДЕНИЕ}
\end{center}
%\refstepcounter{chapter}
\addcontentsline{toc}{chapter}{ВВЕДЕНИЕ}



\textbf{Актуальность}

На протяжении всего прошлого столетия, большие усилия были посвящены исследованию физики
мягкой материи: все их термодинамические свойства на фазовой диаграмме ниже критической точки в настоящее время хорошо известны. С другой стороны, экспериментальные исследования в сверхкритической области были ограничены
до сих пор из-за технических трудностей.
Структурно-динамические исследования, направленные на расширение изучения фазовой диаграммы жидкости
далеко за пределы критической точки играют решающую роль во многих фундаментальных
и прикладных областях исследований, такие как физика конденсированного состояния, планетология, нанотехнологии и управление отходами \cite{WL3}.

Знание зависимости макроскопических параметров системы от потенциала взаимодействия в ней частиц, является открытым вопросом физики мягкой материи. Точное прогнозирование, или хотя бы качественная их оценка, для систем с известным составом и внешними условиями (например, внешними электрическими или магнитными полями), позволят избежать дорогостоящих исследований каждого отдельного вещества. Понимание влияния этих параметров на термодинамику системы, играет важную роль не только с точки зрения фундаментальных знаний, но и прикладных, например, в материаловедении, промышленности и медицине.
Также это открывает возможности для создания новых веществ, удовлетворяющих потребности в определенном фазовом поведении, с нужными температурами плавления или скоростью звука и диффузией.

\newpage

\textbf{Цель бакалаврской квалификационной работы} --
установить связь между дальнодействием притяжения в двумерной системе частиц, взаимодействующих посредством обобщенного потенциала Леннарда--Джонса, c фазовой диаграммой, и параметрами переноса.

\textbf{Задачами работы являются:}
\begin{enumerate}
\item Разработка программного комплекса для расчета явлений переноса в $2D$ системах.
\item Разработка методов определения термодинамических свойств системы по распределениям плотностей. 
\item Усовершенствование метода распознавание фаз и построения фазовых диаграмм.
\item Применение разработанных методов на различных потенциалах взаимодействия.
\item Применение наработок для изучения влияния потенциала взаимодействия на различные параметры системы.
\end{enumerate}
