
\newpage
\begin{center}
\textbf{\large ГЛАВА 1 \\ Роль дальнодействия притяжения в коллективной динамике простых жидкостей}
\end{center}
\refstepcounter{chapter}


\addcontentsline{toc}{chapter}{ГЛАВА 1. Роль дальнодействия притяжения в коллективной динамике простых жидкостей}

\section{Роль отталкивания потенциала взаимодействия}

\section{Роль притяжения потенциала взаимодействия}


\section{Цели и задачи бакалаврской работы}

\textbf{Цель работы} -- установить связь дальнодействия притяжения потенциала и спектров возбуждений с транспортными свойствами жидкостей, а также влияние на скорость нуклеации.

\textbf{Задачи работы:}
\begin{enumerate}
    \item Расчет фазовых диаграмм для 2D и 3D систем частиц, взаимодействующих посредством обобщенного потенциала Леннарда--Джонса с различными степенями притяжения.
    \item Адаптация метода кластеризации данных DBSCAN для изучения молекулярных систем и его сравнение с другими методами.
    \item Расчет и анализ транспортных свойств и коллективных возбуждений на жидкостных бинодалях.
    \item Применение нового метода распознавания фаз для изучения скорости нуклеации в переохлажденных системах Леннарда-Джонса с различным дальнодействием притяжения.
\end{enumerate}
