
\newpage
\begin{center}
\textbf{\large ГЛАВА 1 \\ Роль дальнодействия притяжения в коллективной динамике простых жидкостей}
\end{center}
\refstepcounter{chapter}


\addcontentsline{toc}{chapter}{ГЛАВА 1. Роль дальнодействия притяжения в коллективной динамике простых жидкостей}

\section{Роль отталкивания потенциала взаимодействия}

\section{Роль притяжения потенциала взаимодействия}

\subsection{Влияние дальнодействия потенциала на критическое поведение}

Многие из межмолекулярных сил, играющих центральную роль в обширных областях химии, физики и биологии, имеют дальнодействующую природу.
Хорошо известными примерами являются электростатические взаимодействия, силы поляризации и силы Ван-дер-Ваальса.
Примечательно, что в наших знаниях о критическом поведении, вызванном этими взаимодействиями, все еще имеются значительные недостатки.

Нынешнее понимание критического поведения в системах с такими алгебраически затухающими взаимодействиями в значительной степени основано на расчетах ренормализационной группы.
Показано, что для универсальных критических свойств можно выделить разные режимы, характеризуемые затухающей мощностью взаимодействий.
Ввиду небольшого числа глобальных параметров, определяющих класс универсальности, значительный интерес представляет расположение границ между этими режимами.

В данной главе рассматриваются границы между режимами взаимодействия, которые определеляют класс универсальности. Данный подход специализируется на модели Изинга, $n = 1$, в $d$ измерениях, описываемой редуцированным гамильтонианом

\begin{equation}
\mathcal{H} / k_{\mathrm{B}} T=-K \sum_{\langle i j\rangle} \frac{s_{i} s_{j}}{r_{i j}^{d+\sigma}}
\end{equation}

где спины $s_{k}=\pm 1$ помечены узлом решетки $k$, сумма распространяется на все пары спинов, а связь пар зависит от расстояния $r_{i j}=\left|\vec {r}_{i}-\vec{r}_{j}\right|$ между спинами. Согласно анализу Фишера~\cite{10.1103/PhysRevLett.29.917} классы универсальности параметризованы $\sigma$, и были определены следующие три различных режима: (a) классический режим; верхняя критическая размерность определяется как $d_{u}=2\sigma$, так что критическое поведение типа среднего поля возникает при $\sigma \leq d / 2$. б) промежуточный режим $d/2<\sigma<2$; здесь критические показатели являются непрерывными функциями от $\sigma$. c) режим ближнего действия; для $\sigma\geq 2$ универсальными являются свойства модели с короткодействующими взаимодействиями, например, только между ближайшими соседями; таким образом, можно заметить, что при $d=3$ ван-дер-ваальсовы взаимодействия (затухающие как $1/r^{6}$) действительно лежат довольно близко к границе между режимами (b) и (c).

Хотя общий план этих результатов получил широкое признание, одна часть этой картины стала предметом споров. Это касается ситуации, близкой к $\sigma=2$. В работе~\cite{10.1103/PhysRevLett.29.917} было высказано предположение, что во всем промежуточном режиме (b) показатель корреляционной функции $\eta$ в точности равен $2-\sigma$. С другой стороны, в короткодействующем режиме (в) $\eta$ принимает постоянное (но зависящее от $d$) значение $\eta_{\mathrm{sr}}>0$ для всех $d<4 $, что приводит к скачку разрыва в $\eta$ при степени затухания $\sigma=2$. Хотя это замечательное явление не запрещается термодинамическими аргументами (для которых требуется только $\eta\leq 2+\sigma$ ), оно привлекло значительное внимание в последние десятилетия, и были предприняты различные усилия для повторного исследования соответствующего сценария $R G$~\cite{10.1103/PhysRevB.8.281, 10.1088/0305-4470/22/6/024}. Кроме того, можно отметить, что этот сценарий не охватывает одномерный случай, когда строго известно~\cite{10.1007/BF01654281} отсутствие фазового перехода при $\sigma>1$, а не при $\sigma>2$. Первым к этому вопросу обратился Сак~\cite{10.1103/PhysRevB.8.281}, который указал, что члены высших порядков в уравнениях $R G$, рассмотренных в~\cite{10.1103/PhysRevLett.29.917} генерируют дополнительные короткодействующие взаимодействия в процессе перенормировки, что влияет на конкуренцию между дальнодействующими и короткодействующими неподвижными точками $RG$-преобразования.

Как следствие, при $d<4$ граница между промежуточным и ближним режимами смещается от $\sigma=2$ к $\sigma=2-\tilde{\eta}$, где Разложение $\varepsilon$ для $\tilde{\eta}$ согласуется в низших порядках с разложением ближнего показателя $\eta_{\mathrm{sr}}$. Используя теоретико-полевой подход, Хонконен и Налимов~\cite{10.1088/0305-4470/22/6/024} доказали во всех порядках теории возмущений устойчивость малодействующей неподвижной точки при $\sigma>2-\eta_{\mathrm{sr}}$ и его дальнодействующий аналог для $\sigma<2-\eta_{\mathrm{lr}}$, где $\eta_{\text {lr }}$ — аномальная размерность поля, оцененная при фиксированном дальнодействии точка. Обратите внимание, что эти авторы также указали, что первый результат может быть получен из простых аргументов масштабирования, а второй - нет. Привлекательными аспектами этих результатов являются, во-первых, непрерывная и монотонная $\sigma$-зависимость показателя корреляционной функции (при условии, что $\eta_{\mathrm{Ir}}$ и $\eta_{\mathrm{sr}}$ совпадают при $\sigma=2-\eta_{\mathrm{sr}}$ ), и, во-вторых, что теперь теория достигла согласованности с точными результатами для одномерного случая.



\section{Цели и задачи бакалаврской работы}

\textbf{Цель работы} -- установить связь дальнодействия притяжения потенциала и спектров возбуждений с транспортными свойствами жидкостей, а также влияние на скорость нуклеации.

\textbf{Задачи работы:}
\begin{enumerate}
    \item Расчет фазовых диаграмм для 2D и 3D систем частиц, взаимодействующих посредством обобщенного потенциала Леннарда--Джонса с различными степенями притяжения.
    \item Адаптация метода кластеризации данных DBSCAN для изучения молекулярных систем и его сравнение с другими методами.
    \item Расчет и анализ транспортных свойств и коллективных возбуждений на жидкостных бинодалях.
    \item Применение нового метода распознавания фаз для изучения скорости нуклеации в переохлажденных системах Леннарда-Джонса с различным дальнодействием притяжения.
\end{enumerate}
