\newpage
\begin{center}
  \textbf{\large 1. РОЛЬ ДАЛЬНОДЕЙСТВИЯ ПРИТЯЖЕНИЯ В ПРОСТЫХ ЖИДКОСТЯХ}
\end{center}
\refstepcounter{chapter}
\addcontentsline{toc}{chapter}{1. РОЛЬ ДАЛЬНОДЕЙСТВИЯ ПРИТЯЖЕНИЯ В ПРОСТЫХ ЖИДКОСТЯХ}


\section{Влияние дальнодействия потенциала на критическое поведение}

Многие из межмолекулярных сил, играющих центральную роль в химии, физике и биологии, обладают дальнодействующим потенциалом взаимодействия.
Самые известные примеры: электростатические взаимодействия, поляризационные силы и силы Ван-дер-Ваальса.
Однако в наших знаниях о критическом поведении, вызванном этими взаимодействиями, все еще имеются значительные пробелы.

Понимание критического поведения в системах с такими алгебраически затухающими взаимодействиями в значительной степени основано на расчетах ренормализационной группы.
Доказано, что у критических свойств выделяются разные режимы, которые характерезуются дальностью взаимодействий.
Ввиду небольшого числа параметров, которые определяют класс универсальности, наибольший интерес представляет расположение границ между этими режимами.


Используемый в работе подход основан на модели Изинга, в $d$ измерениях, описываемый редуцированным гамильтонианом
\begin{equation}
  \mathcal{H} / k_{\mathrm{B}} T=-K \sum_{\langle i j\rangle} \frac{s_{i} s_{j}}{r_{i j}^{d+\sigma}}
  \label{eq1}
\end{equation}

где спины $s=\pm 1$, суммирование происходит по всем парам спинов, а взаимодействие пар зависит от расстояния $r_{i j}=\left|\vec {r}_{i}-\vec{r}_{j}\right|$ между спинами. 
Согласно анализу Фишера~\cite{10.1103/PhysRevLett.29.917} классы универсальности параметризованы $\sigma$, так были определены следующие три различных режима: (a) классический режим; (б) промежуточный режим $d/2<\sigma<2$: здесь критические показатели являются непрерывными функциями от $\sigma$; (c) режим ближнего действия: для $\sigma\geq 2$ универсальными являются свойства модели с короткодействующими взаимодействиями, например, только между ближайшими соседями. 
Таким образом, при $d=3$ Ван-дер-Ваальсовы взаимодействия (затухающие как $1/r^{6}$) лежат довольно близко к границе между режимами (b) и (c).

И хотя данное решение получило широкое признание, часть его стала предметом споров. 
Вопрос касается ситуации близкой к $\sigma=2$.
В работе~\cite{10.1103/PhysRevLett.29.917} было высказано предположение, что во всем промежуточном режиме (b) показатель корреляционной функции $\eta$ в точности равен $2-\sigma$. 
С другой стороны, в короткодействующем режиме (в) $\eta$ принимает постоянное (но зависящее от $d$) значение $\eta_{\mathrm{sr}}>0$ для всех $d<4 $, что приводит к разрыву в $\eta$ при степени затухания $\sigma=2$.
Несмотря на то что подобное явление не противоречит термодинамическим законам (для которых требуется только $\eta\leq 2+\sigma$ ), оно привлекло значительное внимание в последние десятилетия, были предприняты усилия для повторного исследования соответствующего подхода~\cite{10.1103/PhysRevB.8.281, 10.1088/0305-4470/22/6/024}.
Кроме того, отметим, что этот подход не охватывает одномерный случай, когда строго известно~\cite{10.1007/BF01654281} отсутствие фазового перехода при $\sigma>1$, а не при $\sigma>2$. 
Первым к этому вопросу обратился Сак~\cite{10.1103/PhysRevB.8.281}, который указал, что рассмотренные в~\cite{10.1103/PhysRevLett.29.917} члены высших порядков в уравнениях генерируют дополнительные короткодействующие взаимодействия в процессе перенормировки.

Как следствие, при $d<4$ граница между промежуточным и ближним режимами смещается от $\sigma=2$ к $\sigma=2-\tilde{\eta}$.
Важными аспектами результатов этих исследований~\cite{10.1088/0305-4470/22/6/024} являются, во-первых, непрерывная и монотонная $\sigma$-зависимость показателя корреляционной функции (при условии, что $\eta_{\mathrm{Ir}}$ и $\eta_{\mathrm{sr}}$ совпадают при $\sigma=2-\eta_{\mathrm{sr}}$ ), и, во-вторых, тот факт, что теперь теория является согласованной с точными результатами для одномерного случая.

Так как основной проблемой является переходные области между промежуточным режимом и режимом ближнего действия, предполагается, что поправки к масштабированию будут сходиться медленно, с постепенным увеличением размера системы.
Данная работа требует моделирования больших систем. 
Используя кластерный алгоритм Монте-Карло~\cite{10.1142/S0129183195000265}, в настоящее время были получены высокоточные данные для достаточно больших размеров системы.

Для представления численных результатов критического показателя вычисляется $\eta$ и кумулянт Биндера~\cite{10.1007/BF01293604} в зависимости от $\sigma$. 
Моделируемые системы задаются на решетках $L\times L$ с периодическими границами и размерами от $L=4$ до $L=1000$. 
Для изучения выбираются двумерные системы с максимально достижимым линейным размером системы.
Заметим, что в частности показатель степени $\eta_{\mathrm{sr}}=\frac{1}{4}$ имеет гораздо большее значение, чем при $d=3\left(\eta_{\mathrm{sr}}=0.037\right)$. 
Данный вывод позволяет заявить, что максимизируется как размер интересующей области $\left\langle 2-\eta_{\mathrm{sr}}, 2\right\rangle$, так и величина предполагаемого скачка $\eta(\sigma)$.
Продолжительность моделирования выбирается таким образом, чтобы (для систем самых больших размеров) достигалась относительная неопределенность в одну тысячную для кумулянта Биндера.
Точная форма парного взаимодействия принимается как:

\begin{equation}
  \tilde{K}(|\vec{r}|)=K \int_{r_{x}-(1 / 2)}^{r_{x}+(1 / 2)} d x \int_{r_{y}-(1 / 2)}^{r_{y}+(1 / 2)} d y \frac{1}{\left(x^{2}+y^{2}\right)^{(d+\sigma) / 2}}
\end{equation}

где $\vec r=\left(r_{x}, r_{y}\right)$ обозначает разность между целыми координатами двух взаимодействующих спинов. 
Отметим, что это взаимодействие, принятое из чисто технических соображений~\cite{10.1142/S0129183195000265}, отличается от взаимодействия в уравнении\ref{eq1} только степенями $r$, убывающими быстрее, чем $r^{-d-\sigma}$.
Критические индексы и границы режимов (а)–(c) не изменятся.

Проблемой в приведённых расчетах является тот факт, что показатели степени коррекции к масштабированию по существу неизвестны и фактически зависят от граничного значения $\sigma$. 
Были предприняты значительные усилия, чтобы охватить все подобные неопределенности в указанных полях для оценки $K_{c}$, $Q$, $\eta$.

Следовательно, как показатель корреляционной функции $\eta$, так и отношение амплитуд четвертого порядка $Q$ принимают свои (универсальные) короткодействующие изинговские значения для $\sigma>2-\eta_{\mathrm{sr}}$. 
Для $\sigma>2$ приведённое выше увтерждение можно продемонстрировать с высокой численной точностью. Для $2-\eta_{\mathrm{sr}}<\sigma<2$ результаты же будут более точными, чтобы исключить переход от дальнодействующего критического поведения к ближнедействующему при $\sigma=2$.
Вместо этого они переходят при $\sigma=2-\eta_{\mathrm{sr}}$. 
Эти результаты прекрасно согласуются с $\eta=2-\sigma$ в промежуточном диапазоне $d/2<\sigma<2-\eta_{\mathrm{sr}}$, подтверждая гипотезу о том, что все вклады высших порядков обращаются в нуль в разложении $\varepsilon^{\prime}$ для $\sigma$ ~\cite{10.1103/PhysRevLett.29.917}.
Отношение амплитуд $Q$ зависит практически линейно от $\sigma$ для $d/2<\sigma<2-\eta_{\mathrm{sr}}$.
Примечательно, что наиболее заметные отклонения от линейности возникают вблизи $\sigma=2-\eta_{\mathrm{sr}}$, в то время как разложение $\varepsilon^{\prime}$ предсказывает сингулярность, подобную квадратному корню, на противоположном конце промежуточного диапазона~\cite{10.1103/PhysRevE.60.7558}.


\section{Влияние дальнодействия потенциала на фазовые диаграммы и плавление}

Понимание фазовых переходов в 2D-системах имеет большое значение в ряде областей, начиная с фотоники и электроники  и заканчивая новыми материалами и биотехнологиями, поскольку знание фазового поведения открывает путь к проектированию систем с желаемыми свойствами. 
Несмотря на многочисленные исследования, основные вопросы в данной области по-прежнему связаны с влиянием конкретного взаимодействия между отдельными частицами на их коллективное поведение. 
Для классических систем одной из простейших моделей, способных воспроизвести поведение веществ, включая газовую, жидкую и твердую фазы, является система Леннарда-Джонса (LJ). 
Модель LJ широко используется для анализа поведения молекулярных, белковых, полимерных, эмульсионных и коллоидных мягких веществ. 
Обобщенный LJ-потенциал (или LJn-m-потенциал, где индексы n и m отвечают за алгебраические ветви отталкивания и притяжения) является подходящей моделью для исследований, направленных на выявление эффектов отталкивания и притяжения в жидкостях, твердых телах и фазовых переходов между ними.

В настоящий момент установлено, что 2D-сценарии плавления зависят от мягкости отталкивания, обеспечивая микроскопические сценарии 2D-плавления, описываемые в работах~\cite{10.3367/ufne.2017.06.038161, 10.3367/ufne.2018.04.038417}, что доказывает теория Березинского-Костерлица-Таулесса-Гальперина-Нельсона-Янга (БКТГНЯ), согласно которой плавление происходит через два непрерывных перехода с промежуточной гексатической фазой с квазидальним ориентационным порядком и ближним трансляционным порядком~\cite{10.1088/0022-3719/6/7/010, 10.1103/physrevlett.41.121, 10.1103/physrevb.19.2457, 10.1103/physrevb.19.1855}, плавление через фазовый переход первого рода, двухстадийное плавление, включающее непрерывный (Березинский-Костерлиц-Таулесс, БКТ) кристаллогексатический фазовый переход и фазовый переход первого рода между гексатической фазой и изотропной жидкостью.
Второй и третий сценарии присущи системам с короткодействующим (жестким) отталкиванием, тогда как первый наблюдался при мягком отталкивании между частицами. 
Установлено, что мягкость отталкивания влияет на сценарии плавления, термодинамику и спектры возбуждения в монослойных системах. 
Однако известно, что роль притяжения в сценарии плавления монослойных систем остается систематически неизученной.

LJ-взаимодействия были одними из первых систем, попытки изучения которых предпринимались для понимания роли притяжения в плавлении. 
Тем не менее, многие опубликованные результаты, рассматривающие критическую точку и сценарий плавления для 2D-кристаллов LJ, не дают исчерпывающего ответа на роль притяжения в данных процессах.
Например, чтобы определить критическую температуру в зависимости от радиуса обрезки потенциала, было выполнено численное моделирование кривой пар-жидкость в ансамбле Гиббса, согласно ~\cite{10.1063/1.460477}.
О противоречивых сценариях плавления треугольного кристалла говорилось в ранних работах~\cite{10.1103/physrevlett.42.1632, 10.1063/1.436526, 10.1103/physrevlett.44.463, 10.1063/1.441901, 10.1103/physrevlett.52.449, 10.1103/physrevb.30.2755}, включая два непрерывных перехода с промежуточной гексатической фазой по теории БКТГНЯ~\cite{10.1103/physrevlett.42.1632} и переход первого рода~\cite{10.1063/1.436526, 10.1103/physrevlett.44.463, 10.1063/1.441901, 10.1103/physrevlett.52.449}.

Благодаря росту вычислительных возможностей моделирование больших систем ($\gtrsim 10^5$ частиц) дало новые результаты по двумерному плавлению кристаллов Леннарда-Джонса и связанных с ними систем.
Моделирование систем с последующим анализом их уравнения состояния и дальнодействующей асимптотики трансляционной корреляционной функции (которая точно обеспечивает предел устойчивости кристалла) позволило однозначно идентифицировать сценарии плавления. 
Например, об изменении сценария плавления говорилось в работе~\cite{10.1103/physreve.99.022145}, где авторы изучали двумерные системы частиц, взаимодействующих посредством обобщенного потенциала Леннарда-Джонса с различными ветвями отталкивания ($\propto 1/r ^{12}$ и $\propto 1/r^{64}$).
Выявлено, что сценарий реализуется через фазовые переходы первого рода при низких температурах и через два непрерывных перехода БКТ при высоких.
Раньше предполагалось, что LJ-система при высоких температурах близка к мягким отталкивающим дискам $1/r^{12}$, но такая экстраполяция на сценарий плавления противоречит результатам приведённого исследования~\cite{10.1103/physrevlett.114.035702}, согласно котррому мягкие диски $1/r^n$ с $n>6$ плавятся по третьему сценарию. 
Предполагалось, что петля Майера-Вуда, присущая переходу первого рода, исчезает при высоких температурах с увеличением размера системы. 
Однако объяснение эффекта конечно-размерным масштабированием кажется неубедительным: с увеличением размера системы петля должна сплющиваться и в конечном итоге приближаться к плато~\cite{10.1103/physreve.87.042134, 10.1103/physreve.59.2659}.

Было установлено, что при низких температурах, где преобладает роль притяжения, все системы плавятся по переходу первого рода за счет подавления гексатической фазы.
При высоких температурах LJ-диски плавятся по третьему сценарию, как и мягкие диски~\cite{10.1103/physrevlett.114.035702}.

Известно, что кристаллы LJ по сравнению с системой Морзе в~\cite{10.1103/physrevb.103.094107} плавятся по третьему сценарию при низких температурах. 
Данный вывод согласуется с~\cite{10.1103/physreve.99.022145}, но противоречит~\cite{10.1103/physrevlett.114.035702}. 
Сценарий БКТГНЯ при высоких температурах был поставлен под сомнение из-за кажущегося исчезновения петли Майера-Вуда, аналога петли Ван-дер-Ваальса в трехмерном случае.
Для мягких взаимодействий Морзе третий сценарий плавления наблюдается для всех температур, рассмотренных в~\cite{10.1103/physrevb.103.094107}, тогда как авторы исследования ожидали наблюдать сценарий БКТГНЯ при более высоких температурах.
Однако, с некоторыми параметрами мягкости потенциала уже при низких температурах, учитывая дальнодействующее притяжение, наблюдались два непрерывных перехода.

Роль притяжения можно проверить экспериментально в коллоидных системах, известных как модельные системы, демонстрирующих широкий спектр ``молекулярно-подобных'' явлений~\cite{book.fernandez, book.ivlev, 10.1016/0370-1573(94)90017-5, 10.1038/natrevmats.2015.11, 10.1039/c9sm01953g}, в частности кристаллизация и плавление~\cite{10.1126/science.1112399, 10.1039/c2sm26473k, 10.1103/physrevlett.82.2721, 10.1103/physrevlett.85.3656, 10.1103/physrevlett.118.088003, 10.1039/c2sm27654b, 10.1126/science.1224763, 10.1038/s41598-021-97124-7}.

Эти коллективные явления визуализируются в реальном времени с пространственным разрешением отдельных частиц.
Дальнодействующее дипольное притяжение $\propto 1/r^3$ в коллоидных системах индуцируется и контролируется in situ с помощью вращающегося в плоскости магнитного поля~\cite{10.1088/0034-4885/76/12/126601, 10.1039/c3sm50306b, 10.1039/c3sm27620a, 10.1103/physrevmaterials.2.025602} или электрического~\cite{10.1088/1367-2630/8/11/267, 10.1063/1.3115641, 10.1021/la2014804, 10.1021/la500178b, 10.1039/c1sm06414b, 10.1038/s41598-017-14001-y} поля.
Используя конически вращающиеся магнитные или электрические поля с магическими углами, может быть создано Ван-Дер-Ваальсово притяжение $\propto 1/r^6$ с "магическими" полями~\cite{10.1021/la500896e, 10.1103/physrevlett.103.228301}.
В последнее время настраиваемые взаимодействия были достигнуты за счет использования пространственных годографов внешнего электрического или магнитного поля~\cite{10.1039/d0sm01046d}, проектирования внутренней структуры~\cite{10.1063/5.0055566} и геометрии~\cite{10.1063/5.0060705} коллоидных частиц.

Моделирование систем частиц производится с помощью обобщенного потенциала Леннарда-Джонса (LJn-m):

\begin{equation}
  U_{n m}(r)=\frac{\epsilon}{n-m}\left[m\left(\frac{\sigma}{r}\right)^{n}-n\left(\frac{\sigma}{r}\right)^{m}\right]
  \label{LJnm}
\end{equation}

где $n$ и $m$ — индексы отталкивающей и притягивающей ветвей соответственно, а $\sigma$ и $\epsilon$ — характерная длина взаимодействия и глубина потенциальной ямы.
Потенциал имеет минимум $-\epsilon$ при $r/\sigma=1$.
В дальнейшем нормируются расстояния и энергии на $\sigma$ и $\epsilon$ соответственно и рассматриваются частицы одинаковой массы $m=1$.

Вблизи критической температуры вычисление плотностей газа и конденсата становится затруднительным из-за растущих флуктуаций плотности в системе.
Тем не менее, следующим образом может быть рассчитано положение критической точки на фазовой диаграмме путем аппроксимации конденсированных и газовых бинодальных ветвей вблизи критической точки:
\begin{equation}
  n_{c}-n_{g} \simeq A \tau^{\beta}, \quad n_{c}+n_{g} \simeq a \tau+2 n_{\mathrm{CP}},
  \label{MACR-eq4}
\end{equation}
где $\tau=T_{\mathrm{CP}}-T$, $T_{\mathrm{CP}}$ и $n_{\mathrm{CP}}$ -- это температура и 
плотность в критической точке соответственно, $\beta$ -- критический индекс, $A$ и $a$ являются параметрами, которые должны быть получены из аппроксимации $n_{\mathrm{CP}}$ и $T_{\mathrm{CP}}$.
Критический индекс $\beta$ зависит от класса универсальности системы, определяемого межчастичным взаимодействием~\cite{10.1103/physrevlett.89.025703}.

Результаты для бинодали конденсат-газ, полученные с помощью метода фазовой идентификации и уравнения состояния, представлены на рисунке~\ref{nmp}.
Цветными кругами обозначены плотности газа, конденсата и их среднее значение для каждого рассмотренного потенциала. 
Сплошные серые линии — области, в которых использовали аппроксимацию для получения значений критической точки с помощью уравнений~\ref{MACR-eq4}. 
Серыми пунктирными линиями показана экстраполяция фазовой диаграммы до критических точек, обозначенных цветными звездочками.

\begin{figure}[!h]
  \begin{center}
    \includegraphics[width=\textwidth]{NMP-Figure4.pdf}
    \caption{Влияние диапазона притяжения на область сосуществования жидкость-газ на фазовой диаграмме: (a) бинодали конденсат-газ для разных потенциалов LJ12-m; круги — точки бинодали и медианы (полученные методом фазовой идентификации), ромбы — точки, полученные из уравнения состояния, серые линии — аппроксимации бинодали, звездочками обозначены критические точки.
      (b) Зависимости тройной и критической температур от индекса притяжения m для взаимодействия LJ12-m, отношение $T_{CP}$/$T_{TP}$ показано на вставке.}
    \label{nmp}
  \end{center}
\end{figure}

Падение диапазона притяжения снижает критическую температуру, а также отношение между температурами критической и тройной точек, как показано на рисунке~\ref{nmp}(b) и соответствующей вставке.
С увеличением $m$ двухфазная область сужается в сторону меньших плотностей, а отношение между критической и тройной температурами приближается к единице.
Для LJ-взаимодействия ($m = 6$) полученная критическая температура $T_c=(0,51 . . . 0,52)$ (в зависимости от метода оценки) согласуется с предыдущими результатами $T_c = 0.515 \pm 0.002$  для LJ-потенциала.

В данном разделе был проведен обзор эволюции фазовых диаграмм и сценариев плавления двумерных систем частиц, взаимодействующих через обобщенный потенциал Леннарда-Джонса с разным диапазоном притяжения, в то время как ветвь отталкивания зафиксирована.

Переход жидкость-газ изучается с помощью анализа уравнения состояния и метода фазовой идентификации.
Результаты, полученные двумя упомянутыми методами, хорошо согласуются друг с другом.
Плавление при высоких температурах и высоких плотностях в системе мягких сфер $1/r_{12}$ происходит согласно третьему сценарию. 
Однако при низких температурах плавления в системах с $m = 6$, $9$ и $11$ было выявлено изменение сценария плавления от третьего к переходу первого порядка (без гексатической фазы).
Обнаружено, что температура изменения сценариев смещается в сторону более низких температур с увеличением диапазона притяжения, что соответствует уменьшению $m$. 
Анализ случая $m = 9 (LJ12-9)$  показал, что для короткодействующего притяжения наблюдается третий сценарий плавления.

Однако на данный момент не существует теории, которая предсказывала бы поведение транспортных свойств и коллективных возбуждений в зависимости от дальнодействия притяжения.
В связи с этим формулируются следующие цели и задачи настоящей работы.

\section{Цели и задачи магистерской работы}

\textbf{Цель работы} -- установить связь дальнодействия притяжения потенциала взаимодействия и спектров возбуждений с транспортными свойствами жидкостей, а также влияние на скорость нуклеации.

\textbf{Задачи работы:}
\begin{enumerate}
\item Расчет фазовых диаграмм для 2D и 3D систем частиц, взаимодействующих посредством обобщенного потенциала Леннарда-Джонса с различными степенями притяжения.
\item Адаптация метода кластеризации данных DBSCAN для изучения молекулярных систем и его сравнение с другими методами.
\item Расчет и анализ транспортных свойств и коллективных возбуждений на жидкостных бинодалях.
\item Применение нового метода распознавания фаз для изучения скорости нуклеации в переохлажденных системах Леннарда-Джонса с различным дальнодействием притяжения.
\end{enumerate}
