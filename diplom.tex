\documentclass[%
candidate, % тип документа
subf, % подключить и настроить пакет subfig для вложенной нумерации рисунков
%href, % подключить и настроить пакет hyperref
%colorlinks=true % цветные гиперссылки
times % шрифт Times как основной
%,fixint=false % отключить прямые знаки интегралов
]{disser}

%\renewcommand{\rmdefault}{ftm}
\usepackage{tempora}

\usepackage[
a4paper, mag=1000,
left=3cm, right=1cm, top=2cm, bottom=2cm, headsep=0.7cm, footskip=1cm
]{geometry}


\usepackage[T2A]{fontenc}
\usepackage[utf8]{inputenc}
\usepackage[english,russian]{babel}

%\usepackage{paratype}
%\defaultfontfeatures{Ligatures={TeX},Renderer=Basic} 
%\setmainfont[Ligatures={TeX,Historic}]{Times New Roman}
\usepackage{pdfpages}
\ifpdf\usepackage{epstopdf}\fi

\usepackage{dcolumn}
\usepackage{bm}
\usepackage{hyperref}
\usepackage{color}
\usepackage{epstopdf}
\usepackage{amsmath}
\usepackage{amssymb}
\usepackage{cite}
\usepackage{multirow}
\usepackage{afterpage}
\usepackage[font={normal}]{caption}
%\usepackage{setspace}
\usepackage{amsmath} % align
\usepackage[onehalfspacing]{setspace}% 1,5 интервал
%\usepackage{biblatex}
\usepackage{fancyhdr} % пакет для установки колонтитулов
\pagestyle{fancy} % смена стиля оформления страниц
\fancyhf{} % очистка текущих значений
\fancyfoot[C]{\thepage} % установка верхнего колонтитула
\renewcommand{\headrulewidth}{0pt} % убрать разделительную линию

\captionsetup{format=hang,labelsep=period}

% Использовать полужирное начертание для векторов
\let\vec=\mathbf

% Включать подсекции в оглавление
\setcounter{tocdepth}{2}

\graphicspath{{images/}}


\pagestyle{footcenter}
\chapterpagestyle{footcenter}


\begin{document}
\includepdf[pages={1-1}]{chapters/Titul_vkr.pdf}
% Содержание

\newpage
\begin{center}
    \textbf{\large АННОТАЦИЯ}
\end{center}


Основная цель дипломной работы заключается в разработке автоматизированной информационной системы расписания МТУСИ.
Данный комплекс программ предоставляет возможность пользователям просматривать расписание занятий университета.

Для достижения цели работы были поставлены следующие задачи: 
\begin{enumerate}
    \item Разработка GraphQl API для доступа к расписанию.
    \item Разработка парсера расписания из таблиц exel в электронный формат.
    \item Разработка мобильного приложения Android и iOS.
    \item Разработка ICS API для интеграции с приложениями календарей и разработка чат-бота в ВК и Телеграм.
\end{enumerate}

В процессе исследования была изучена документация ПО.
В результате были разработаны мобильные приложения с расписаниями МТУСИ для Android и iOS, и
открытое GraphQL API для доступа к расписанию любых программ.

Все участники учебного процесса МТУСИ теперь могут пользоваться мобильными приложениями, 
а желающие разработать собственный сервис могут использовать GraphQL API для получения актуального расписания.

Таким образом, данная работа демонстрирует возможности автоматизации информационных процессов в университете и
значительно упрощает процесс поиска расписания на любой день для любого участника учебного процесса.

\onehalfspacing
\setcounter{page}{2}

\newpage
\renewcommand{\contentsname}{\centerline{\large СОДЕРЖАНИЕ}}
\tableofcontents

\newpage
\begin{center}
\textbf{\large ВВЕДЕНИЕ}
\end{center}
\addcontentsline{toc}{chapter}{ВВЕДЕНИЕ}


\textbf{Актуальность}
Актуальность выпускной квалификационной работы по разработке автоматизированной информационной системы расписания МТУСИ проявляется в нескольких аспектах.
\begin{enumerate}
    \item Проект предлагает использование новых технологий для оптимизации образовательного процесса, что является важным шагом в развитии современного образования. 
    Разработанный комплекс программ значительно сократит время, затрачиваемое на поиск расписания занятий, и обеспечит участникам учебного процесса 
    удобный и быстрый доступ к расписанию.
    \item Актуальность проекта так же обюсуловлена глобальным трендом активного использования мобильных устройств. 
    Разработанные мобильные приложения для iOS и Android позволят пользователям легко получать доступ к расписанию, 
    а также интегрировать его в персональные календари. Это удовлетворит потребности современного общества, где мобильные технологии играют все более важную роль в повседневной жизни.
\end{enumerate}

\textbf{Научной новизна} данного проекта состоит в применении новейших технологий, методологий и подходов к разработке программного обеспечения. 
Разработка GraphQl API, парсера расписания из таблиц Excel, микросервисной архитектуры и Progressive Web App представляют собой инновационные решения, 
которые ранее не применялись в данной области. 
Такое применение новейших технологий позволит интегрировать существующие методы ведения учебного процесса 
с современными способами получения информации, открывая новые возможности для оптимизации образовательных процессов и повышения их эффективности.

\newpage

\textbf{Цель выпускной квалификационной работы} -- Разработать комплекс программ для удобного доступа к расписанию занятий университета МТУСИ.
Комплекс программ позволит многократно упростить и ускорить процесс поиска расписания на любой день для любого участника учебного процесса.
Расписание можно будет искать как по учебной группе, преподавателю, учебной дисциплине, кафедре, дню и времени,
так и по любому сочетанию этих фильтров.
Доступ к расписанию будет осуществляться через мобильные приложения iOS и android, и интеграцию в персональный календарь.

\textbf{Задачи выпускной квалификационной работы:}
\begin{enumerate}
    \item Разработка GraphQl API для доступа к расписанию.
    \item Разработка парсера расписания из таблиц exel в электронный формат.
    \item Разработка мобильных приложений Android и iOS.
    \item Разработка ICS API для интеграции с приложениями календарей.
    \item Разработка микросервисной архитектуры для распределения нагрузки.
    \item Разработка PWA (Progressive Web App) для удобного использования расписания на компьютере.
\end{enumerate}
\newpage
\newpage
\begin{center}
  \textbf{\large 1. РОЛЬ ДАЛЬНОДЕЙСТВИЯ ПРИТЯЖЕНИЯ В ПРОСТЫХ ЖИДКОСТЯХ}
\end{center}
\refstepcounter{chapter}
\addcontentsline{toc}{chapter}{1. РОЛЬ ДАЛЬНОДЕЙСТВИЯ ПРИТЯЖЕНИЯ В ПРОСТЫХ ЖИДКОСТЯХ}


\section{Влияние дальнодействия потенциала на критическое поведение}

Многие из межмолекулярных сил, играющих центральную роль в химии, физике и биологии, обладают дальнодействующим потенциалом взаимодействия.
Самые известные примеры: электростатические взаимодействия, поляризационные силы и силы Ван-дер-Ваальса.
Однако в наших знаниях о критическом поведении, вызванном этими взаимодействиями, все еще имеются значительные пробелы.

Понимание критического поведения в системах с такими алгебраически затухающими взаимодействиями в значительной степени основано на расчетах ренормализационной группы.
Доказано, что у критических свойств выделяются разные режимы, которые характерезуются дальностью взаимодействий.
Ввиду небольшого числа параметров, которые определяют класс универсальности, наибольший интерес представляет расположение границ между этими режимами.


Используемый в работе подход основан на модели Изинга, в $d$ измерениях, описываемый редуцированным гамильтонианом
\begin{equation}
  \mathcal{H} / k_{\mathrm{B}} T=-K \sum_{\langle i j\rangle} \frac{s_{i} s_{j}}{r_{i j}^{d+\sigma}}
  \label{eq1}
\end{equation}

где спины $s=\pm 1$, суммирование происходит по всем парам спинов, а взаимодействие пар зависит от расстояния $r_{i j}=\left|\vec {r}_{i}-\vec{r}_{j}\right|$ между спинами. 
Согласно анализу Фишера~\cite{10.1103/PhysRevLett.29.917} классы универсальности параметризованы $\sigma$, так были определены следующие три различных режима: (a) классический режим; (б) промежуточный режим $d/2<\sigma<2$: здесь критические показатели являются непрерывными функциями от $\sigma$; (c) режим ближнего действия: для $\sigma\geq 2$ универсальными являются свойства модели с короткодействующими взаимодействиями, например, только между ближайшими соседями. 
Таким образом, при $d=3$ Ван-дер-Ваальсовы взаимодействия (затухающие как $1/r^{6}$) лежат довольно близко к границе между режимами (b) и (c).

И хотя данное решение получило широкое признание, часть его стала предметом споров. 
Вопрос касается ситуации близкой к $\sigma=2$.
В работе~\cite{10.1103/PhysRevLett.29.917} было высказано предположение, что во всем промежуточном режиме (b) показатель корреляционной функции $\eta$ в точности равен $2-\sigma$. 
С другой стороны, в короткодействующем режиме (в) $\eta$ принимает постоянное (но зависящее от $d$) значение $\eta_{\mathrm{sr}}>0$ для всех $d<4 $, что приводит к разрыву в $\eta$ при степени затухания $\sigma=2$.
Несмотря на то что подобное явление не противоречит термодинамическим законам (для которых требуется только $\eta\leq 2+\sigma$ ), оно привлекло значительное внимание в последние десятилетия, были предприняты усилия для повторного исследования соответствующего подхода~\cite{10.1103/PhysRevB.8.281, 10.1088/0305-4470/22/6/024}.
Кроме того, отметим, что этот подход не охватывает одномерный случай, когда строго известно~\cite{10.1007/BF01654281} отсутствие фазового перехода при $\sigma>1$, а не при $\sigma>2$. 
Первым к этому вопросу обратился Сак~\cite{10.1103/PhysRevB.8.281}, который указал, что рассмотренные в~\cite{10.1103/PhysRevLett.29.917} члены высших порядков в уравнениях генерируют дополнительные короткодействующие взаимодействия в процессе перенормировки.

Как следствие, при $d<4$ граница между промежуточным и ближним режимами смещается от $\sigma=2$ к $\sigma=2-\tilde{\eta}$.
Важными аспектами результатов этих исследований~\cite{10.1088/0305-4470/22/6/024} являются, во-первых, непрерывная и монотонная $\sigma$-зависимость показателя корреляционной функции (при условии, что $\eta_{\mathrm{Ir}}$ и $\eta_{\mathrm{sr}}$ совпадают при $\sigma=2-\eta_{\mathrm{sr}}$ ), и, во-вторых, тот факт, что теперь теория является согласованной с точными результатами для одномерного случая.

Так как основной проблемой является переходные области между промежуточным режимом и режимом ближнего действия, предполагается, что поправки к масштабированию будут сходиться медленно, с постепенным увеличением размера системы.
Данная работа требует моделирования больших систем. 
Используя кластерный алгоритм Монте-Карло~\cite{10.1142/S0129183195000265}, в настоящее время были получены высокоточные данные для достаточно больших размеров системы.

Для представления численных результатов критического показателя вычисляется $\eta$ и кумулянт Биндера~\cite{10.1007/BF01293604} в зависимости от $\sigma$. 
Моделируемые системы задаются на решетках $L\times L$ с периодическими границами и размерами от $L=4$ до $L=1000$. 
Для изучения выбираются двумерные системы с максимально достижимым линейным размером системы.
Заметим, что в частности показатель степени $\eta_{\mathrm{sr}}=\frac{1}{4}$ имеет гораздо большее значение, чем при $d=3\left(\eta_{\mathrm{sr}}=0.037\right)$. 
Данный вывод позволяет заявить, что максимизируется как размер интересующей области $\left\langle 2-\eta_{\mathrm{sr}}, 2\right\rangle$, так и величина предполагаемого скачка $\eta(\sigma)$.
Продолжительность моделирования выбирается таким образом, чтобы (для систем самых больших размеров) достигалась относительная неопределенность в одну тысячную для кумулянта Биндера.
Точная форма парного взаимодействия принимается как:

\begin{equation}
  \tilde{K}(|\vec{r}|)=K \int_{r_{x}-(1 / 2)}^{r_{x}+(1 / 2)} d x \int_{r_{y}-(1 / 2)}^{r_{y}+(1 / 2)} d y \frac{1}{\left(x^{2}+y^{2}\right)^{(d+\sigma) / 2}}
\end{equation}

где $\vec r=\left(r_{x}, r_{y}\right)$ обозначает разность между целыми координатами двух взаимодействующих спинов. 
Отметим, что это взаимодействие, принятое из чисто технических соображений~\cite{10.1142/S0129183195000265}, отличается от взаимодействия в уравнении\ref{eq1} только степенями $r$, убывающими быстрее, чем $r^{-d-\sigma}$.
Критические индексы и границы режимов (а)–(c) не изменятся.

Проблемой в приведённых расчетах является тот факт, что показатели степени коррекции к масштабированию по существу неизвестны и фактически зависят от граничного значения $\sigma$. 
Были предприняты значительные усилия, чтобы охватить все подобные неопределенности в указанных полях для оценки $K_{c}$, $Q$, $\eta$.

Следовательно, как показатель корреляционной функции $\eta$, так и отношение амплитуд четвертого порядка $Q$ принимают свои (универсальные) короткодействующие изинговские значения для $\sigma>2-\eta_{\mathrm{sr}}$. 
Для $\sigma>2$ приведённое выше увтерждение можно продемонстрировать с высокой численной точностью. Для $2-\eta_{\mathrm{sr}}<\sigma<2$ результаты же будут более точными, чтобы исключить переход от дальнодействующего критического поведения к ближнедействующему при $\sigma=2$.
Вместо этого они переходят при $\sigma=2-\eta_{\mathrm{sr}}$. 
Эти результаты прекрасно согласуются с $\eta=2-\sigma$ в промежуточном диапазоне $d/2<\sigma<2-\eta_{\mathrm{sr}}$, подтверждая гипотезу о том, что все вклады высших порядков обращаются в нуль в разложении $\varepsilon^{\prime}$ для $\sigma$ ~\cite{10.1103/PhysRevLett.29.917}.
Отношение амплитуд $Q$ зависит практически линейно от $\sigma$ для $d/2<\sigma<2-\eta_{\mathrm{sr}}$.
Примечательно, что наиболее заметные отклонения от линейности возникают вблизи $\sigma=2-\eta_{\mathrm{sr}}$, в то время как разложение $\varepsilon^{\prime}$ предсказывает сингулярность, подобную квадратному корню, на противоположном конце промежуточного диапазона~\cite{10.1103/PhysRevE.60.7558}.


\section{Влияние дальнодействия потенциала на фазовые диаграммы и плавление}

Понимание фазовых переходов в 2D-системах имеет большое значение в ряде областей, начиная с фотоники и электроники  и заканчивая новыми материалами и биотехнологиями, поскольку знание фазового поведения открывает путь к проектированию систем с желаемыми свойствами. 
Несмотря на многочисленные исследования, основные вопросы в данной области по-прежнему связаны с влиянием конкретного взаимодействия между отдельными частицами на их коллективное поведение. 
Для классических систем одной из простейших моделей, способных воспроизвести поведение веществ, включая газовую, жидкую и твердую фазы, является система Леннарда-Джонса (LJ). 
Модель LJ широко используется для анализа поведения молекулярных, белковых, полимерных, эмульсионных и коллоидных мягких веществ. 
Обобщенный LJ-потенциал (или LJn-m-потенциал, где индексы n и m отвечают за алгебраические ветви отталкивания и притяжения) является подходящей моделью для исследований, направленных на выявление эффектов отталкивания и притяжения в жидкостях, твердых телах и фазовых переходов между ними.

В настоящий момент установлено, что 2D-сценарии плавления зависят от мягкости отталкивания, обеспечивая микроскопические сценарии 2D-плавления, описываемые в работах~\cite{10.3367/ufne.2017.06.038161, 10.3367/ufne.2018.04.038417}, что доказывает теория Березинского-Костерлица-Таулесса-Гальперина-Нельсона-Янга (БКТГНЯ), согласно которой плавление происходит через два непрерывных перехода с промежуточной гексатической фазой с квазидальним ориентационным порядком и ближним трансляционным порядком~\cite{10.1088/0022-3719/6/7/010, 10.1103/physrevlett.41.121, 10.1103/physrevb.19.2457, 10.1103/physrevb.19.1855}, плавление через фазовый переход первого рода, двухстадийное плавление, включающее непрерывный (Березинский-Костерлиц-Таулесс, БКТ) кристаллогексатический фазовый переход и фазовый переход первого рода между гексатической фазой и изотропной жидкостью.
Второй и третий сценарии присущи системам с короткодействующим (жестким) отталкиванием, тогда как первый наблюдался при мягком отталкивании между частицами. 
Установлено, что мягкость отталкивания влияет на сценарии плавления, термодинамику и спектры возбуждения в монослойных системах. 
Однако известно, что роль притяжения в сценарии плавления монослойных систем остается систематически неизученной.

LJ-взаимодействия были одними из первых систем, попытки изучения которых предпринимались для понимания роли притяжения в плавлении. 
Тем не менее, многие опубликованные результаты, рассматривающие критическую точку и сценарий плавления для 2D-кристаллов LJ, не дают исчерпывающего ответа на роль притяжения в данных процессах.
Например, чтобы определить критическую температуру в зависимости от радиуса обрезки потенциала, было выполнено численное моделирование кривой пар-жидкость в ансамбле Гиббса, согласно ~\cite{10.1063/1.460477}.
О противоречивых сценариях плавления треугольного кристалла говорилось в ранних работах~\cite{10.1103/physrevlett.42.1632, 10.1063/1.436526, 10.1103/physrevlett.44.463, 10.1063/1.441901, 10.1103/physrevlett.52.449, 10.1103/physrevb.30.2755}, включая два непрерывных перехода с промежуточной гексатической фазой по теории БКТГНЯ~\cite{10.1103/physrevlett.42.1632} и переход первого рода~\cite{10.1063/1.436526, 10.1103/physrevlett.44.463, 10.1063/1.441901, 10.1103/physrevlett.52.449}.

Благодаря росту вычислительных возможностей моделирование больших систем ($\gtrsim 10^5$ частиц) дало новые результаты по двумерному плавлению кристаллов Леннарда-Джонса и связанных с ними систем.
Моделирование систем с последующим анализом их уравнения состояния и дальнодействующей асимптотики трансляционной корреляционной функции (которая точно обеспечивает предел устойчивости кристалла) позволило однозначно идентифицировать сценарии плавления. 
Например, об изменении сценария плавления говорилось в работе~\cite{10.1103/physreve.99.022145}, где авторы изучали двумерные системы частиц, взаимодействующих посредством обобщенного потенциала Леннарда-Джонса с различными ветвями отталкивания ($\propto 1/r ^{12}$ и $\propto 1/r^{64}$).
Выявлено, что сценарий реализуется через фазовые переходы первого рода при низких температурах и через два непрерывных перехода БКТ при высоких.
Раньше предполагалось, что LJ-система при высоких температурах близка к мягким отталкивающим дискам $1/r^{12}$, но такая экстраполяция на сценарий плавления противоречит результатам приведённого исследования~\cite{10.1103/physrevlett.114.035702}, согласно котррому мягкие диски $1/r^n$ с $n>6$ плавятся по третьему сценарию. 
Предполагалось, что петля Майера-Вуда, присущая переходу первого рода, исчезает при высоких температурах с увеличением размера системы. 
Однако объяснение эффекта конечно-размерным масштабированием кажется неубедительным: с увеличением размера системы петля должна сплющиваться и в конечном итоге приближаться к плато~\cite{10.1103/physreve.87.042134, 10.1103/physreve.59.2659}.

Было установлено, что при низких температурах, где преобладает роль притяжения, все системы плавятся по переходу первого рода за счет подавления гексатической фазы.
При высоких температурах LJ-диски плавятся по третьему сценарию, как и мягкие диски~\cite{10.1103/physrevlett.114.035702}.

Известно, что кристаллы LJ по сравнению с системой Морзе в~\cite{10.1103/physrevb.103.094107} плавятся по третьему сценарию при низких температурах. 
Данный вывод согласуется с~\cite{10.1103/physreve.99.022145}, но противоречит~\cite{10.1103/physrevlett.114.035702}. 
Сценарий БКТГНЯ при высоких температурах был поставлен под сомнение из-за кажущегося исчезновения петли Майера-Вуда, аналога петли Ван-дер-Ваальса в трехмерном случае.
Для мягких взаимодействий Морзе третий сценарий плавления наблюдается для всех температур, рассмотренных в~\cite{10.1103/physrevb.103.094107}, тогда как авторы исследования ожидали наблюдать сценарий БКТГНЯ при более высоких температурах.
Однако, с некоторыми параметрами мягкости потенциала уже при низких температурах, учитывая дальнодействующее притяжение, наблюдались два непрерывных перехода.

Роль притяжения можно проверить экспериментально в коллоидных системах, известных как модельные системы, демонстрирующих широкий спектр ``молекулярно-подобных'' явлений~\cite{book.fernandez, book.ivlev, 10.1016/0370-1573(94)90017-5, 10.1038/natrevmats.2015.11, 10.1039/c9sm01953g}, в частности кристаллизация и плавление~\cite{10.1126/science.1112399, 10.1039/c2sm26473k, 10.1103/physrevlett.82.2721, 10.1103/physrevlett.85.3656, 10.1103/physrevlett.118.088003, 10.1039/c2sm27654b, 10.1126/science.1224763, 10.1038/s41598-021-97124-7}.

Эти коллективные явления визуализируются в реальном времени с пространственным разрешением отдельных частиц.
Дальнодействующее дипольное притяжение $\propto 1/r^3$ в коллоидных системах индуцируется и контролируется in situ с помощью вращающегося в плоскости магнитного поля~\cite{10.1088/0034-4885/76/12/126601, 10.1039/c3sm50306b, 10.1039/c3sm27620a, 10.1103/physrevmaterials.2.025602} или электрического~\cite{10.1088/1367-2630/8/11/267, 10.1063/1.3115641, 10.1021/la2014804, 10.1021/la500178b, 10.1039/c1sm06414b, 10.1038/s41598-017-14001-y} поля.
Используя конически вращающиеся магнитные или электрические поля с магическими углами, может быть создано Ван-Дер-Ваальсово притяжение $\propto 1/r^6$ с "магическими" полями~\cite{10.1021/la500896e, 10.1103/physrevlett.103.228301}.
В последнее время настраиваемые взаимодействия были достигнуты за счет использования пространственных годографов внешнего электрического или магнитного поля~\cite{10.1039/d0sm01046d}, проектирования внутренней структуры~\cite{10.1063/5.0055566} и геометрии~\cite{10.1063/5.0060705} коллоидных частиц.

Моделирование систем частиц производится с помощью обобщенного потенциала Леннарда-Джонса (LJn-m):

\begin{equation}
  U_{n m}(r)=\frac{\epsilon}{n-m}\left[m\left(\frac{\sigma}{r}\right)^{n}-n\left(\frac{\sigma}{r}\right)^{m}\right]
  \label{LJnm}
\end{equation}

где $n$ и $m$ — индексы отталкивающей и притягивающей ветвей соответственно, а $\sigma$ и $\epsilon$ — характерная длина взаимодействия и глубина потенциальной ямы.
Потенциал имеет минимум $-\epsilon$ при $r/\sigma=1$.
В дальнейшем нормируются расстояния и энергии на $\sigma$ и $\epsilon$ соответственно и рассматриваются частицы одинаковой массы $m=1$.

Вблизи критической температуры вычисление плотностей газа и конденсата становится затруднительным из-за растущих флуктуаций плотности в системе.
Тем не менее, следующим образом может быть рассчитано положение критической точки на фазовой диаграмме путем аппроксимации конденсированных и газовых бинодальных ветвей вблизи критической точки:
\begin{equation}
  n_{c}-n_{g} \simeq A \tau^{\beta}, \quad n_{c}+n_{g} \simeq a \tau+2 n_{\mathrm{CP}},
  \label{MACR-eq4}
\end{equation}
где $\tau=T_{\mathrm{CP}}-T$, $T_{\mathrm{CP}}$ и $n_{\mathrm{CP}}$ -- это температура и 
плотность в критической точке соответственно, $\beta$ -- критический индекс, $A$ и $a$ являются параметрами, которые должны быть получены из аппроксимации $n_{\mathrm{CP}}$ и $T_{\mathrm{CP}}$.
Критический индекс $\beta$ зависит от класса универсальности системы, определяемого межчастичным взаимодействием~\cite{10.1103/physrevlett.89.025703}.

Результаты для бинодали конденсат-газ, полученные с помощью метода фазовой идентификации и уравнения состояния, представлены на рисунке~\ref{nmp}.
Цветными кругами обозначены плотности газа, конденсата и их среднее значение для каждого рассмотренного потенциала. 
Сплошные серые линии — области, в которых использовали аппроксимацию для получения значений критической точки с помощью уравнений~\ref{MACR-eq4}. 
Серыми пунктирными линиями показана экстраполяция фазовой диаграммы до критических точек, обозначенных цветными звездочками.

\begin{figure}[!h]
  \begin{center}
    \includegraphics[width=\textwidth]{NMP-Figure4.pdf}
    \caption{Влияние диапазона притяжения на область сосуществования жидкость-газ на фазовой диаграмме: (a) бинодали конденсат-газ для разных потенциалов LJ12-m; круги — точки бинодали и медианы (полученные методом фазовой идентификации), ромбы — точки, полученные из уравнения состояния, серые линии — аппроксимации бинодали, звездочками обозначены критические точки.
      (b) Зависимости тройной и критической температур от индекса притяжения m для взаимодействия LJ12-m, отношение $T_{CP}$/$T_{TP}$ показано на вставке.}
    \label{nmp}
  \end{center}
\end{figure}

Падение диапазона притяжения снижает критическую температуру, а также отношение между температурами критической и тройной точек, как показано на рисунке~\ref{nmp}(b) и соответствующей вставке.
С увеличением $m$ двухфазная область сужается в сторону меньших плотностей, а отношение между критической и тройной температурами приближается к единице.
Для LJ-взаимодействия ($m = 6$) полученная критическая температура $T_c=(0,51 . . . 0,52)$ (в зависимости от метода оценки) согласуется с предыдущими результатами $T_c = 0.515 \pm 0.002$  для LJ-потенциала.

В данном разделе был проведен обзор эволюции фазовых диаграмм и сценариев плавления двумерных систем частиц, взаимодействующих через обобщенный потенциал Леннарда-Джонса с разным диапазоном притяжения, в то время как ветвь отталкивания зафиксирована.

Переход жидкость-газ изучается с помощью анализа уравнения состояния и метода фазовой идентификации.
Результаты, полученные двумя упомянутыми методами, хорошо согласуются друг с другом.
Плавление при высоких температурах и высоких плотностях в системе мягких сфер $1/r_{12}$ происходит согласно третьему сценарию. 
Однако при низких температурах плавления в системах с $m = 6$, $9$ и $11$ было выявлено изменение сценария плавления от третьего к переходу первого порядка (без гексатической фазы).
Обнаружено, что температура изменения сценариев смещается в сторону более низких температур с увеличением диапазона притяжения, что соответствует уменьшению $m$. 
Анализ случая $m = 9 (LJ12-9)$  показал, что для короткодействующего притяжения наблюдается третий сценарий плавления.

Однако на данный момент не существует теории, которая предсказывала бы поведение транспортных свойств и коллективных возбуждений в зависимости от дальнодействия притяжения.
В связи с этим формулируются следующие цели и задачи настоящей работы.

\section{Цели и задачи магистерской работы}

\textbf{Цель работы} -- установить связь дальнодействия притяжения потенциала взаимодействия и спектров возбуждений с транспортными свойствами жидкостей, а также влияние на скорость нуклеации.

\textbf{Задачи работы:}
\begin{enumerate}
\item Расчет фазовых диаграмм для 2D и 3D систем частиц, взаимодействующих посредством обобщенного потенциала Леннарда-Джонса с различными степенями притяжения.
\item Адаптация метода кластеризации данных DBSCAN для изучения молекулярных систем и его сравнение с другими методами.
\item Расчет и анализ транспортных свойств и коллективных возбуждений на жидкостных бинодалях.
\item Применение нового метода распознавания фаз для изучения скорости нуклеации в переохлажденных системах Леннарда-Джонса с различным дальнодействием притяжения.
\end{enumerate}

\newpage
\newpage
\begin{center}
  \textbf{\large 2. МОДЕЛЬ ПЛАВЛЕНИЯ ПЕРЕГРЕТЫХ КРИСТАЛЛОВ}
\end{center}
\refstepcounter{chapter}
\addcontentsline{toc}{chapter}{2. МОДЕЛЬ ПЛАВЛЕНИЯ ПЕРЕГРЕТЫХ КРИСТАЛЛОВ}


\section{Введение}

Плавление -- один из наиболее изученных фазовых переходов, важных для атомных, молекулярных, коллоидных и белковых систем.
Однако в настоящее время не существует микроскопических экспериментально доступных критериев, которые можно было бы использовать для надежного отслеживания эволюции системы при переходе, критериев, дающих возможность понять физику зародышеобразования при плавлении и эволюции фронта плавления.
Чтобы решить эту проблему, была разработана теоретическая модель в рамках теории среднего поля с использованием нового локального параметра порядка (экспериментально измеримого) -- нормированного среднеквадратичного смещения между частицами в соседних ячейках Вороного.
Предложенная модель протестирована при помощи компьютерного моделирования при различных режимах динамики частиц (броуновским и ньютоновским).
В результате обнаружено, что данная модель обеспечивает превосходное описание эволюции системы при пересечении линии плавления.

Явление плавления широко распространено в природе и может быть обнаружено, начиная от атомных и молекулярных до белковых и коллоидных систем и выходя далеко за рамки материаловедения, поэтому формулировке критериев плавления в микроскопическом масштабе уделяется значительное внимание уже около 100 лет.
Один из наиболее широко используемых микроскопических подходов принадлежит Линдеману~\cite{lindemann1910}, в частности отметим, его переосмысление Гилварри~\cite{10.1103/physrev.102.308}: то, что сейчас известно как критерий Линдемана -- это утверждение, что плавление происходит, когда среднеквадратичное смещение атомов от их положения достигает определенной доли (обычно 0.1-0.15) межатомного расстояния.
Популярность критерия обусловлена его простотой и интуитивно понятным способом применения, но есть и существенные недостатки, включая низкую точность в прогнозировании температуры плавления и отсутствие явного учета жидкого состояния~\cite{10.1098/rspa.1991.0068}.
Впоследствии был введен ряд критериев, которые можно проследить до работы Гилварри~\cite{10.1063/1.1426419}, в особенности, с развитием современных экспериментальных и вычислительных методов, обеспечивающих доступ к смещениям атомов.
Но детали структурных изменений, главным образом в отношении этих критериев, остаются неуловимыми, а ряд ключевых явлений, куда входят механизм зародышеобразования, кинетики фронта плавления и поведения системы вблизи точки плавления, остаются малоизученными.


\section{Материалы и методы}

\subsection{$\lambda^2$-Параметр: Локальная мера беспорядка}
\label{SSMF-AppA}

В работе~\cite{10.1021/acs.jpcc.7b09317} для характеризации локальной разупорядоченности и идентификации конденсированных, жидких или твердых фаз, в конденсируемых системах был предложен подход, основанный на анализе ячеек Вороного -- элементарной площади (''объема''), занимаемой каждой частицей в системе.
В рамках данного подхода на первом этапе система разбивается на ячейки Вороного, чтобы вычислить следующий параметр:
\begin{equation}
  \label{SSMF-eq1}
  \sigma_{i} =\frac{1}{a_i N_{ni}}\sqrt{\sum_{j<k}^{N_{ni}}{(r_{ij}-r_{ik})^2}/2}, \quad r_{ij}=|\mathbf{r}_i-\mathbf{r}_j|,
\end{equation}
где $\mathbf{r}_i$ -- радиус-вектор $i$-ой частицы, $N_{ni}$ -- количество соседних ячеек, $a_i = \sqrt{S_i/\pi}$ -- характерный радиус, $S_i$ -- площадь ячейки Вороного.
На втором этапе для подавления сильных локальных тепловых флуктуаций проводится усреднение с соседними ячейками, которые имеют общую грань (сторона в 2D случае) с ячейкой частицы $i$ \cite{10.1021/acs.jpcc.7b09317}:
\begin{equation}
  \label{SSMF-eq2}
  \lambda_{i} = \frac{1}{N_{ni}+1}\left(\sigma_{i}+\sum_{j=1}^{N_{ni}}{\sigma_{j}}\right).
\end{equation}
В результате получается стандартное отклонение $\lambda_i^2$ расстояний между соседними частицами в \emph{физически малом объеме} в окрестности $i$-й частицы.
Важно отметить, что $\lambda_i^2$ одинаково хорошо применимо для характеризации как твердой, так и жидкой фаз системы, поскольку этот параметр связан с локальным беспорядком в физически малом объеме~\cite{10.1021/acs.jpcc.7b09317}.
В кристаллах $\lambda^2$ связано с параметром Линдемана для соседних частиц~\cite{10.1016/0375-9601(85)90617-6}, так как $\lambda^2 \propto \sigma_\|^2$, где $\sigma_\|^2$ -- продольная компонента среднеквадратичного смещения ближайших частиц.


\subsection{Детали моделирования МД}
\label{SSMF-AppC}

Проведено МД-моделирование кристаллов в условиях ланжевеновской динамики.
В качестве характерного примера рассмотрена система частиц, взаимодействующих по обратному степенному закону (IPL18):
\begin{equation}
  \label{SSMF-eq3}
  \varphi(r) = \epsilon a \left(\frac{\sigma}{r}\right)^{18},
\end{equation}
где $\epsilon$ и $\sigma$ -- сила и характерный масштаб отталкивания соответственно,
а параметр $ a = 2.365 $ введен для удобства моделирования скачкообразного изменения диаметра частиц.
Была использована нормированная температура $ T/ \epsilon \rightarrow T $, расстояние $ r/ \sigma \rightarrow r $, плотность частиц $\rho\sigma^3/m\rightarrow n$ и время $t\sqrt{\epsilon/m\sigma^2} \rightarrow t$ ($m$ -- масса частицы).

Для анализа плавления перегретого кристалла выполнено МД-моделирование системы, состоящей из $ N = 7.2 \times 10 ^ 4 $ частиц в $NVT$ ансамбле при $n=0.867$ и $T=1$.
В исходном состоянии частицы располагались в ГЦК решетке, ориентированной таким образом, чтобы плоскость (111) совпадала с горизонталью.
Размеры области моделирования в $ x $, $ y $ и $ z $ - направлениях выбраны с учетом $ L_x / L_z \approx 20.4 $ и $ L_y / L_z \approx 21.3 $.
Временной шаг взят равным $ \Delta t = 7.4 \times 10 ^ {- 4} \sqrt {m \sigma ^ 2 / \epsilon}$.
Расчеты методом молекулярной динамики проводились в открытом программном пакете LAMMPS.
Моделирование проводилось в 2 этапа: (i) система моделировалась на протяжении $ 10 ^ 5 $ временных шагов с $ a = 7.224 $ для достижения состояния равновесия; (ii) значение $a$ фиксировалось равным $ 2.365 $ и проводилось дополнительное моделирование на $ 4 \times 10 ^ 5 $ шагов для анализа плавления в кристалле.

\section{Результаты}
\subsection{Самосогласованная модель среднего поля эволюции $\lambda^2$ - поля}

Примеры кристаллических и жидких структур показаны на Рис.~\ref{SSMF-Figure1}(a) и \ref{SSMF-Figure1}(b) соответственно.
Белые точки -- частицы; ячейки Вороного показаны сплошными серыми линиями и окрашены в соответствии со значением параметра $\lambda^2$ -- нормированным среднеквадратичным смещением между частицами в соседних ячейках Вороного~\cite{10.1021/acs.jpcc.7b09317}.
В кристаллах $\lambda^2$ связано с параметром Линдемана для соседних частиц~\cite{10.1016/0375-9601(85)90617-6}.
Данное условие проиходит из того, что $\lambda^2\propto \sigma_ \| ^ 2 $, где $ \sigma_ \| ^ 2 $ -- продольная составляющая среднеквадратичного смещения ближайших частиц.
Кроме того, $ \sigma_ \| ^ 2 $ играет важную роль в вычислении первого корреляционного пика в кристаллах \cite{10.1063/1.4869863, 10.1063/1.4926945, 10.1088/0953-8984/28/23/235401, 10.1039/c7sm02429k, 10.1063/1.5116176}.
После плавления кристаллическая решетка разрушается, но, несмотря на диффузию частиц, разбиение Вороного все еще применимо в жидкости.

В случае систем с отталкиванием, рост $\lambda^2$ обеспечивается (i) повышением температуры или (ii) уменьшением плотности.
В то время как первый механизм имеет ключевую роль в системах с мягким отталкиванием между частицами (например, в мягких кристаллах при низких температурах $\lambda^2\propto T $), последний является определяющим в системах типа твердых сфер (таких как коллоиды NIPAm), коллективная динамика которых определяется объемной долей частиц.
В обоих случаях $\lambda^2$ счиатется параметром порядка, и в этих терминах плавление представляет собой переход от состояний с малым $\lambda^2$ (кристалл) к состояниям с большим $\lambda^2$ (жидкость).

Чтобы получить самосогласованную модель эволюции $ \lambda^2$-поля,
необходимо рассмотреть слабо неоднородное пространственное поле $\lambda^2$.
Параметр $\lambda ^ 2$ неконсервативен, следовательно его эволюция определяется нестационарным уравнением Гинзбурга-Ландау (или моделью Ланжевена)~\cite{book.desai}:
\begin{equation}
  \label{SSMF-eq4}
  \frac{\partial \lambda^2}{\partial t} = -\Gamma \frac{\delta \mathcal{F}}{\delta \lambda^2} + \varepsilon^{1/2}\xi(t,\mathbf{r}),
\end{equation}
где $\Gamma$ -- обобщенная вязкость, $ \mathcal{F} $ -- функционал свободной энергии системы, $\langle \xi(t,\mathbf{r})\xi(t',\mathbf{r}')\rangle = \delta(t-t')\delta(\mathbf{r}-\mathbf{r}')$ и $\varepsilon = 2k_BT\Gamma$.
Последнее слагаемое в~\eqref{SSMF-eq2} описывает тепловые флуктуации поля $\lambda^2$.

\begin{figure}[!t]
  \centering
  \includegraphics[width=100mm]{SSMF-Figure1.pdf}
  \caption{\textbf{Схематичное изображение к предлагаемой самосогласованной $\lambda^2$~-~модели:}
    (a) и (b) снимки системы в кристаллическом и жидком состоянии (взяты из МД моделирования).
    Ячейки Вороного раскрашены в соответствии со значениями параметра $ \lambda^2$.
    Панели (c) и (d) схематично иллюстрируют (синие линии) зависимость свободной энергии $ F (\lambda ^ 2) $ (в однородной системе) и обобщенную мощность $ Q (\lambda ^ 2) $, сопряженную с полем $\lambda^2$.
    Пунктирные красные линии изображают ступенчатое приближение \eqref{SSMF-eq5} и \eqref{SSMF-eq7}.
  }
  \label{SSMF-Figure1}
\end{figure}


\begin{figure*}[!t]
  \centering
  \includegraphics[width=\linewidth]{SSMF-Figure3.pdf}
  \caption{\textbf{Автомодельный $ \lambda^2$-профиль распространяющегося фронта плавления в перегретом кристалле, наблюдаемый при МД-моделировании:}
    (a) - (c) Последовательные снимки системы, где круги представляют собой частицы, окрашенные в соответствии со значением $\lambda^2$.
    (d) Эволюция поля $\lambda^2(r, t)$ в радиальном направлении (1), показанном на (a).
    (e) $\lambda^2(\tau)$ -- профиль распространяющегося фронта плавления при росте зародышей.
    Красные символы -- экспериментальные точки, красная сплошная линия -- теоретическая аппроксимация \eqref{SSMF-eq9}.
    Синие символы представляют собой долю ячеек Вороного с 6-ю соседями в плоскости анализа, а резкое падение указывает на разрушение кристаллической структуры.}
  \label{SSMF-Figure3}
\end{figure*}

Функционал свободной энергии $\mathcal{F}[\lambda^2] = \int{d\mathbf{r}\;F[\lambda^2]}$ в квадратичном приближении может быть представлен в виде:
\begin{equation}
  \label{SSMF-eq5}
  F[\lambda^2] = F_{\mathrm{1,2}}^{(0)}+\frac{1}{2}A_{1,2}\left(\lambda^2-\lambda_{1,2}^2\right)^2 + \frac{1}{2}\alpha_{1,2}\left(\nabla\lambda^2\right)^2,
\end{equation}
где $F_{1,2}^{(0)}$ -- энергия однородного состояния ($1$ или $2$), $A$ и $\alpha$ -- положительные коэффициенты разложения \cite{book.desai}, а индексы $ 1 $ и $ 2 $ соответствуют кристаллическому или жидкому состоянию при $\lambda^2 \lessgtr \lambda_\ast^2$ соответственно.
Параметр $\lambda_\ast^2 $ -- это порог, связанный с обобщенным критерием плавления типа Линдемана, предполагается, что $ F_{\mathrm{1}} ^ {(0)}> F_{\mathrm{2}} ^ {(0)}$ для рассматриваемого случая.

При помощи уравнений~\eqref{SSMF-eq4} и \eqref{SSMF-eq5} можно получить:
\begin{equation}
  \label{SSMF-eq6}
  \frac{\partial \lambda^2}{\partial t} = \chi_{1,2} \nabla^2\lambda^2 + Q(\lambda^2) +  \varepsilon^{1/2}\xi(t,\mathbf{r}),
\end{equation}
где $ \chi_{1,2} = \alpha_{1,2} \Gamma$ характеризует диффузию $\lambda^2$,
а $ Q (\lambda ^ 2) $ -- обобщенный источник $ \lambda^2$-поля,
\begin{equation}
  \label{SSMF-eq7}
  Q(\lambda^2) =
  \left\{
    \begin{array}{ll}
      -\gamma_{1}\left(\lambda^2-\lambda_{1}^2\right), & \lambda^2  < \lambda_\ast^2;\\
      -\gamma_{2}\left(\lambda^2-\lambda_{2}^2\right), & \lambda^2 > \lambda_\ast^2,\\
    \end{array}
  \right.
\end{equation}
где $\gamma_{1,2} = \Gamma A_{1,2}$.
Уравнение \eqref{SSMF-eq6} демонстрирует аналогию с эволюцией температуры в химически реактивных средах \cite{10.1088/0004-637x/805/1/59} и совпадает с уравнением для кинетической температуры, которое исследовалось в работах~\cite{10.1103/physreve.96.043201, 10.1103/physreve.97.043206, 10.1103/physreve.100.023203} при анализе распространяющихся фронтов неравновесного плавления в однослойных пылевых плазменных кристаллах.

Энергия \eqref{SSMF-eq5} для однородного случая и соответствующая обобщенная мощность $ Q (\lambda ^ 2) $ показаны на Рис.~\ref{SSMF-Figure1}(с) и \ref{SSMF-Figure1}(d).
Как видно на Рис.~\ref{SSMF-Figure1}(d), система может существовать долгое время в окрестности устойчивых состояний с $\lambda^2= \lambda_{1,2} ^ 2 $, тогда как пороговое значение $\lambda^2=\lambda_\ast^2$ соответствует неустойчивой точке.
Ниже показано, что решение уравнения~\eqref{SSMF-eq6} объясняет два важных явления, изучаемых в настоящем исследовании:
(i) распространение автомодельных фронтов плавления в перегретых кристаллах, образованных частицами, движущимися в броуновском или ньютоновском динамическом режиме, и (ii) бифуркационное поведение различных $\lambda^2$ -флуктуаций (зародышей плавления) в перегретом кристалле.

Если пренебречь влиянием теплового шума и полагать кривизну фронта плавления незначительной при его распространении, то это означает, что $ \epsilon \simeq 0 $ и в уравнении~\eqref{SSMF-eq6} можно записать $ \nabla ^ 2 = \partial ^ 2 / \partial r ^ 2 $.
В этом случае, самоподобный профиль (бегущей волны плавления) описывается функцией $\lambda^2(t-r/v_\mathrm{fr})\equiv \lambda^2(\tau)$ (где $v_\mathrm{fr}$ -- скорость фронта), которая подчиняется уравнению:
\begin{equation}
  \label{SSMF-eq8}
  \frac{\chi_{1,2}}{v_{\mathrm{fr}}^2} \frac{d^2 \lambda^2}{d\tau^2} -\frac{d \lambda^2}{d \tau} -\gamma_{1,2}(\lambda^2-\lambda_{1,2}^2) =0,
\end{equation}
учитывая, что $\lambda^2(\tau) $ и его производная $ d\lambda^2/ d \tau $ должны быть непрерывными в точке $ \tau = 0 $, где $\lambda^2=\lambda_\ast^2$.
Полученное уравнение идентично уравнению, которое возникает в задаче неравновесного плавления в комплексных (пылевых) плазменных кристаллах, следовательно, решение уравнения~\eqref{SSMF-eq8} аналогично \cite{10.1103/physreve.96.043201, 10.1103/physreve.100.023203}:
\begin{equation}
  \label{SSMF-eq9}
  \frac{\lambda^2(\tau)-\lambda^2_1}{\lambda^2_\ast-\lambda^2_1}=
  \left\{
    \begin{array}{ll}
      e^{p_1 \tau}, & \tau < 0;\\
      1+\left(1-e^{-p_2\tau}\right)p_1/p_2 , & \tau > 0,\\
    \end{array}
  \right. \\
\end{equation}
где $ p_{1,2} = \left(\sqrt {1 + 4 \gamma_{1,2} \chi_{1,2} / v_\mathrm{fr} ^ 2} \pm 1 \right) v_\mathrm{fr} ^ 2/2 \chi_{1,2} $ -- показатели в экспоненциальных ветвях решения до и после фронта плавления.
В пределе $\tau \gg 1$, $\lambda^2(\tau) \rightarrow \lambda_2^2$, откуда следует, что
$\left(\lambda_2^2-\lambda^2_1\right)/\left(\lambda^2_\ast-\lambda^2_1\right) = (1+p_1/p_2)$.
Скорость фронта плавления и показатели $p_{1,2} $ (неизвестные \emph{априори}) сложным образом определяются диффузией $\lambda^2$, спецификой межчастичных взаимодействий и различием химических потенциалов на границе жидкость-твердое тело~\cite{10.1038/ncomms7942}.


\subsection{Прямое наблюдение автомодельного профиля стационарных фронтов плавления в МД симуляции}
\label{SSMF-Results-MD}

Распространение фронтов плавления -- медленный процесс по сравнению с характерным временем движения отдельных частиц.
Это означает, что описание в терминах медленно флуктуирующего $\lambda^2$-поля должно быть применимым как в коллоидах, демонстрирующих броуновский режим движения отдельных частиц, так и в системах с ланжевеновской динамикой.
Чтобы подтвердить, что все ключевые особенности, наблюдаемые в случае коллоидных систем, присутствуют и в атомарных кристаллах, было выполнено моделирование аналогичного процесса методом МД с термостатом Ланжевена и слабым затуханием.
При параметрах выполненного МД-моделирования плотность системы при плавлении и кристаллизации (в безразмерных единицах) составляет $ n_m = 0,93 $ и $ n_f = 0,88 $, соответственно~\cite{10.1080/00268979500100911}.
Отсюда следует, что ступенчатое изменение диаметра частиц в проведенных расчетах можно оценить как $ (n_f / n) ^ {1/3} -1 \simeq 0.5 \% $ ($ n = 0.867 $), что позволяет вычислить падение эффективной объемной доли частиц относительно ее значения, соответствующего плавлению:
$\Delta \phi  \simeq (n_m/n)^{1/3}-1\simeq 2.4\%$.
Полученное значение близко к режиму промежуточного перегрева, обсуждаемому в работе~\cite{10.1038/ncomms7942}.


Результаты проведенного МД-моделирования автомодельных фронтов \\ плавления в перегретом кристалле частиц с IPL18 взаимодействием представлены на Рис.~\ref{SSMF-Figure3}.
Полученные значения $\lambda^2$ в кристаллическом, жидком и пороговом состояниях равны $\lambda_1^2 \simeq 0.01$, $\lambda_2^2 \simeq 0.07$, и $\lambda_\ast^2 \simeq 0.035$, соответственно.
Значения $\lambda_\ast^2$, полученные в МД моделированиях хорошо согласуются с критерием Линдемана для ближайших соседей \cite{10.1016/0375-9601(85)90617-6}.


\section{Заключение главы}

Поведение поля $\lambda^2$ было проанализировано с использованием нестационарного уравнения Гинзбурга-Ландау с тепловым шумом и источниками.
Показано, что разработанная модель демонстрирует существенно нелинейное поведение, в то время как слагаемые в уравнениях имеют ясный физический смысл в контексте анализа плавления кристаллов.
Кроме того, будучи по своей сути микроскопической, предложенная модель позволяет с высокой степенью детализации изучать зародышеобразование в различных режимах перегрева (в зависимости от величины теплового шума) и эволюцию реалистичных жидких зародышей, которые могут принимать самые разные сложные формы.

Процесс зародышеобразования в перегретых кристаллах, кинетика образования и роста жидких зародышей и структура устойчивых фронтов плавления являются центральными проблемами в понимании плавления кристаллов.
Представленные результаты -- существенный шаг вперед, предоставляя простой и эффективный инструмент для изучения процессов зародышеобразования и плавления в перегретых кристаллах различной природы.

\newpage
\newpage
\begin{center}
  \textbf{\large 3. ДИФФУЗИЯ НА ЖИДКОСТНЫХ БИНОДАЛЯХ: ВЛИЯНИЕ ДАЛЬНОДЕЙСТВИЯ СИЛЫ ПРИТЯЖЕНИЯ}
\end{center}
\refstepcounter{chapter}
\addcontentsline{toc}{chapter}{3. ДИФФУЗИЯ НА ЖИДКОСТНЫХ БИНОДАЛЯХ: ВЛИЯНИЕ ДАЛЬНОДЕЙСТВИЯ СИЛЫ ПРИТЯЖЕНИЯ}


В данной главе используется моделирование молекулярной динамики для расчета фазовых диаграмм обобщенных систем Леннарда-Джонса с различными показателями притяжения.
Оценены коэффициенты диффузии и подвижности (обратной диффузии) и проанализированы спектры коллективного возбуждения на жидких бинодалиях. 
Отмечено, что зависимость коэффициента подвижности от температуры является линейной в широком диапазоне температур, а ее наклон увеличивается с увеличением показателя притяжения.
При отклонении подвижности от линейной зависимости дисперсионные соотношения коллективных возбуждений жидкости переходят от осцилирующего к монотонному виду.

\section{Роль диффузии в науке и технике}
\label{MACR-SecIntroduction}

Диффузия играет решающую роль в различных процессах переноса массы, начиная с науки и техники и заканчивая живой природой.
Диффузия выступает ключевым механизмом в биологических процессах~\cite{10.1016/j.bbagen.2013.09.037, 10.1038/s41598-018-22643-9}, а также в кинетике химических реакций.
Знание механизмов диффузии позволит добиться значительного прогресса в новых биотехнологиях и медицине, решить важные проблемы химической и фармакологической промышленности~\cite{10.1002/3527602836}.

Процесс диффузии очень хорошо изучен в газах и твердых телах.
Например, в кристаллических системах~\cite{10.1016/0079-6816(95)00039-2}, что связано с ее практической ценностью в металлургии для легирования~\cite{10.1016/s0924-0136(96)02826-9, 10.1016/j.actamat.2015.10.010, 10.1134/s1063783411110308} и эксплуатации полупроводниковой электроники~\cite{10.1103/physrevlett.84.4220, 10.1016/j.physrep.2009.10.003}.

В данной главе, используя метод молекулярной динамики (МД), моделируются обобщенные системы Леннарда-Джонса с различными показателями притяжения.
Рассчитаны температурные зависимости подвижности частиц (коэффициента обратной диффузии) на бинодали жидкость-газ. 
Здесь же рассмотрена связь между диффузией, дальнодействием межчастичного притяжения и свойствами коллективных возбуждений в простых жидкостях.

\section{Методы}
\label{MACR-SecMethods}

\subsection{Расчет фазовых диаграмм методами молекулярной динамики}
\label{MACR-SubSecMD}

В этой главе анализируются транспортные свойства и их связь с коллективными модами на жидких бинодальях.
Для систем, взаимодействующих через обобщенный потенциал Леннарда-Джонса (LJ$n$-$m$):
\begin{equation}
  U_{n-m}(r)=4 \varepsilon\left[\left(\frac{\sigma}{r}\right)^{n}-\left(\frac{\sigma}{r}\right)^{m}\right]
  \label{MACR-eq1}
\end{equation}
где $\epsilon$ и $\sigma$ -- характерные масштабы энергии и длины соответственно.
На протяжении всей статьи используются приведенные единицы измерения температуры $ T/ \epsilon \rightarrow T $, расстояния $ r/ \sigma \rightarrow r $ и плотности $ \rho \sigma ^ 3 \rightarrow n$.


Были рассмотрены потенциалы LJ$12$-$4$, LJ$12$-$5$, LJ$12$-$6$ и LJ$16$-$6$.
Чтобы сравнить полученные результаты для LJ$n$-$m$ с результатами для системы, в которой взаимодействия не являются сферически-симметричными, были также смоделировали этан~\cite{10.1021/acs.jced.6b01036}.
В выбранной модели молекула этана рассматривается как пара жестко связанных радикалов CH$_3$, взаимодействующих с радикалами других молекул через потенциал~\cite{10.1021/acs.jced.6b01036}:
\begin{equation}
  U_{\rm {ethane}}(r) = \tilde \varepsilon\left[\left(\frac{\sigma}{r}\right)^{16}-\left(\frac{\sigma}{r}\right)^{6}\right],
  \label{MACR-eq2}
\end{equation}
где $\tilde\varepsilon = 0,69396$ ккал/моль и $\sigma = 3,783$\AA.

Все МД-симуляции были выполнены в ансамбле NVT (N, V и T --количество частиц, объем системы и температура соответственно) с периодическими граничными условиями с использованием пакета моделирования \\  LAMMPS~\cite{10.1006/jcph.1995.1039}.
В первую очередь были рассчитаны линии бинодали~\cite{10.1021/jp806127j, 10.1021/jp1117213}.
Исходное состояние системы формировалось в два этапа: (i) кубическая область моделирования заполнялась равновесным кристаллом (в нашем случае ГЦК) из $N$ частиц с плотностью, соответствующей близкому к нулю давлению; 
(ii) область моделирования была расширена в направлении осей $x$ так, чтобы окончательная средняя плотность системы $\rho_a$ стала равной значениям, указанным в таблице~\ref{MACR-Table1}.
Результирующее начальное состояние показано на рис.~\ref{MACR-Figure1}(а). 
Затем температура системы линейно увеличивалась от $T_{start}$ до $T_{stop}$ в течение $n_{step}$ шагов моделирования с временным шагом $\Delta t$.
Конденсированная фаза в какой-то момент начинает испаряться, образуя сосуществование газа и конденсата, если температура ниже критической, как показано на рис.~\ref{MACR-Figure1}(b).
Принципиально то, что полученное таким образом состояние системы почти всегда имеет границы фаз, ортогональные оси $x$.
В результате плотности $\rho_g$ и $\rho_c$ газовой и конденсированной фаз соответственно могут быть рассчитаны путем подгонки профиля плотности $\rho(x)$ выражением~\cite{10.1021/jp806127j, 10.1021/jp1117213}:

\begin{equation}
  \rho(x)=\frac{\rho_{l}+\rho_{g}}{2}-\frac{\rho_{l}-\rho_{g}}{2} \tanh \left(\frac{|x|-L}{\delta}\right),
  \label{MACR-eq3}
\end{equation}
где $L$ — половина длины области моделирования, занимаемой жидкой фазой, а $\delta$ — характерная ширина границы раздела.
Пример профиля плотности системы и его аппроксимация уравнением~\eqref{MACR-eq3} показаны на рис.~\ref{MACR-Figure1}(c) гистограммой и красной линией соответственно.
Параметры моделирования для рассмотренных моделей сведены в табл.~\ref{MACR-Table1}.

\begin{table}[]
  \centering
  \begin{tabular}{|lllllcl|}
    \hline
    \multicolumn{1}{|l|}{Potential} & \multicolumn{1}{l|}{$\rho_a$} & \multicolumn{1}{l|}{$r_c$} & \multicolumn{1}{l|}{$T_{start}$} & \multicolumn{1}{l|}{$T_{stop}$} & \multicolumn{1}{l|}{$n_{step}$}                       & $\Delta t$                          \\ \hline
    \multicolumn{7}{|c|}{Значения в безразмерных единицах:}                                                                                                                                                                                                         \\ \hline
    \multicolumn{1}{|l|}{LJ12-4}    & \multicolumn{1}{l|}{0.25}     & \multicolumn{1}{l|}{15.0}  & \multicolumn{1}{l|}{1.0}         & \multicolumn{1}{l|}{5.5}        & \multicolumn{1}{c|}{\multirow{4}{*}{$3 \times 10^6$}} & \multirow{4}{*}{$5 \times 10^{-4}$} \\ \cline{1-5}
    \multicolumn{1}{|l|}{LJ12-5}    & \multicolumn{1}{l|}{0.25}     & \multicolumn{1}{l|}{10.0}  & \multicolumn{1}{l|}{0.8}         & \multicolumn{1}{l|}{2.4}        & \multicolumn{1}{c|}{}                                 &                                     \\ \cline{1-5}
    \multicolumn{1}{|l|}{LJ12-6}    & \multicolumn{1}{l|}{0.35}     & \multicolumn{1}{l|}{8.0}   & \multicolumn{1}{l|}{0.5}         & \multicolumn{1}{l|}{1.4}        & \multicolumn{1}{c|}{}                                 &                                     \\ \cline{1-5}
    \multicolumn{1}{|l|}{LJ16-6}    & \multicolumn{1}{l|}{0.31}     & \multicolumn{1}{l|}{8.0}   & \multicolumn{1}{l|}{0.8}         & \multicolumn{1}{l|}{1.6}        & \multicolumn{1}{c|}{}                                 &                                     \\ \hline
    \multicolumn{7}{|c|}{Единицы измерения СИ:}                                                                                                                                                                                                                     \\ \hline
    \multicolumn{1}{|l|}{Ethane}    & \multicolumn{1}{l|}{$0.22\mathrm{\frac{g}{cm^3}}$}     & \multicolumn{1}{l|}{$25\text{\AA}$}    & \multicolumn{1}{l|}{$80\,\mathrm{K}$}          & \multicolumn{1}{l|}{$320\,\mathrm{K}$}        & \multicolumn{1}{l|}{$2 \times 10^6$}                  & $2\,\mathrm{\text{фс}}$                                   \\ \hline
  \end{tabular}
  \caption{Параметры, используемые в МД-моделировании для бимодальных расчетов: где $\rho$ — средняя плотность системы, $r_c$ — радиус отсечки, $T_{start}$ и $T_{stop}$ — начальная и конечная температуры моделирования, соответственно, $n_{step}$ — количество шагов моделирования, а $\Delta t$ — временной шаг.}
  \label{MACR-Table1}
\end{table}


\begin{figure}[!t]
  \centering
  \includegraphics[width=150mm]{MACR-Figure1.png}
  \caption{(a) Система частиц для расчета фазовой диаграммы.
    Система частиц с потенциалом взаимодействия LJ12-6 при температуре $T=1.13$ в виде плоского слоя.
    (b) Профиль плотности системы вдоль оси $x$.
    Область с высокой плотностью представляет собой конденсат, с низкой -- газ.
    Темно-красная линия представляет собой аппроксимацию профиля плотности уравнением~\eqref{MACR-eq3}.}
  \label{MACR-Figure1}
\end{figure}


\begin{figure}[!t]
  \centering
  \includegraphics[width=150mm]{MACR-Figure2.pdf}
  \caption{Фазовая диаграмма системы LJ12-6.
    Оранжевым и синим цветами обозначены символы плотности газа и конденсата соответственно, полученные путем подгонки данных МД по уравнению~\eqref{MACR-eq3}.
    Зеленые символы -- медиана $\rho_m=(\rho_g+\rho_c)/2$.
    Сплошная красная линия соответствует уравнению~\eqref{MACR-eq4}.
    Тройные и критические точки -- синие и красные звездочки соответственно.
  }
  \label{MACR-Figure2}
\end{figure}

Вблизи критической температуры расчет плотности газа и жидкости становится затруднительным из-за усиленных флуктуаций плотности.
Однако положение критической точки на фазовой диаграмме можно вычислить, аппроксимируя жидкостную и газообразную бинодальные ветви вблизи критической точки выражением~\ref{MACR-eq4}.

В трехмерии критический индекс $\beta_c = 0,5$ для потенциала LJ$12-4$, тогда как $\beta_c = 0,325$ для LJ$12$-$5$, LJ$12$-$6$, LJ$16$-$6$ и этана, согласно предыдущим результатам~\cite{10.1021/acs.jced.6b01036,10.1021/jp9072137,10.1103/physrevlett.89.025703}.

Пример полученных бинодалей для LJ12-6 и их аппроксимации уравнением~\eqref{MACR-eq4} показаны на рис.~\ref{MACR-Figure2}.
Обратите внимание, что на конденсированной бинодали имеется явный излом (см. рис.~\ref{MACR-Figure2}), который указывает на падение плотности при плавлении и соответствует положению тройной точки.
Полученные значения $A$ и $a$ из уравнения~\eqref{MACR-eq4}, а также значения плотности и температуры критических и тройных точек для рассматриваемых систем представлены в таблице~\ref{MACR-Table2}.

\begin{figure}[!t]
  \centering
  \includegraphics[width=\linewidth]{MACR-Figure3.pdf}
  \caption{(а) Фазовые диаграммы рассматриваемых систем. 
    Фазовые диаграммы рассчитывались методом двухфазного моделирования, который описан в разделе~\ref{MACR-SecMethods}.
    Цветные точки обозначают рассчитанные бинодали, треугольники -- срединные точки.
    Сплошные серые кривые показывают диапазон температур, используемый для аппроксимации и определения параметров в уравнении ~\eqref{MACR-eq4}.
    Штриховые серые кривые соответствуют экстраполированным биноидам.
    (б) Температурная зависимость подвижности частиц.
    Подвижность частиц была рассчитана на жидких бинодальях с использованием метода, описанного в разделе~\ref{MACR-SecMethods}.
    Точки, соответствующие экстраполированным бинодалим, отмечены серым цветом. 
    Прямые линии -- линейная аппроксимация подвижности.
    На вставке показана расчетная подвижность метана.}
  \label{MACR-Figure3}
\end{figure}

\subsection{Расчет диффузии и спектров на бинодали}

Далее для расчета подвижности на конденсированной бинодали моделировались системы с плотностью и температурой, взятыми из полученных фазовых диаграмм.
Обобщенные систем Леннарда-Джонса с $N = 4,0 \times 10 ^ 3$ моделировались с шагом по времени $1,5 \times 10 ^ 5$.
Для этана использовались $N = 1,065 \times 10 ^ 4 $ молекул и проведены моделирования с шагом по времени $7.0 \times 10^5 $.
Для релаксации системы использовались первые $ 5.0 \times 10 ^ 4 $ временных шагов для обобщенных LJ-систем и $ 5.0 \times 10 ^ 5 $ для этана.
Остальные параметры были такими же, как и при расчете фазовых диаграмм.

Коэффициент самодиффузии $D$ определялся по среднеквадратичному отклонению частиц:
\begin{equation}
  \sigma^2(t) = \sum\limits_{\alpha = 1}^{N} (r_{\alpha}(t) - r_{\alpha}(0))^2 / N, \quad \sigma^2(t) = 6Dt,
  \label{MACR-eq5}
\end{equation}
где $\sigma$ — среднеквадратичное отклонение, а $t$ — время.
Подвижность $\mu$ связана с коэффициентом диффузии соотношением Эйнштейна
\begin{equation}
  \mu = \frac{D}{T},
  \label{MACR-eq6}
\end{equation}
где $T$ — температура системы.

Наконец, спектры возбуждения были получены с использованием обработки тока скорости~\cite{10.1063/1.5050708}:
\begin{equation}
  C_{L, T}(\mathbf{q}, \omega)=\int dt e^{i \omega t} \text{Re} \left\langle\mathbf{j}_{L, T}(\mathbf{q}, t) \mathbf{j}_{L, T}(-\mathbf{q}, 0)\right\rangle,
  \label{MACR-eq7}
\end{equation}
где ${\bf k}$ и $\omega$ — волновой вектор и частота,
$\mathbf{j}_{L}=\mathbf{q}(\mathbf{j} \cdot \mathbf{q} ) / q^{2}$ и $\mathbf{j}_{T}=(\mathbf{j \cdot e_{\perp})e_{\perp}}$ — продольная ($L$) и поперечная ($T$) компоненты тока частиц,\\
$\mathbf{j}(\mathbf{q}, t)=N^{-1} \sum_{s} \mathbf{v}_{s}(t) \ exp \left(i \mathbf{q} \mathbf{r}_{s}(t)\right)$ и $\mathbf{v}_{s}(t)=\dot{\mathbf{r} }_{s}(t)$ — скорость $s$-й частицы.
Суммирование ведется по всем $N$ частицам в системе.
Усреднение по каноническому ансамблю обозначается $\langle\cdots\rangle$. 
Анализ $C_{L, T}(\mathbf{q}, \omega)$ проводился с помощью методов из работы~\cite{10.1038/s41598-019-46979-y}, что позволило получить дисперсионные соотношения продольной и поперечной мод.

МД-моделирование для расчета спектров возбуждения отличается от моделирования для подвижности только длительностью временного шага. Для LJ$12$-$4$ и LJ$16$-$6$ шаг по времени был выбран как $\Delta t = 1 \times 10 ^ {-4} \sqrt {m \sigma ^ 2 / \epsilon}$, а для LJ$12$-$5$ и LJ$12$-$6$ -- $\Delta t = 5 \times 10 ^ {-4} \sqrt {m \sigma ^ 2 / \epsilon}$.

\begin{figure}[!t]
  \centering
  \includegraphics[width=160mm]{MACR-Figure4.pdf}
  \caption{(a) Температурная зависимость подвижности системы LJ$12$-$6$ вдоль жидкостной бинодали.
    Температуры, при которых рассчитывались спектры возбуждения, указаны черными стрелками.
    (b) - (f) спектры возбуждения LJ$12$-$6$ систем.
    Спектры рассчитывались путем анализа скорости течения (уравнение~\eqref{MACR-eq7}) так же, как в Ref.~\cite{10.1038/s41598-019-46979-y}.
    Красный цвет соответствует гибридным модам, серый — результатам анализа отдельных мод~\cite{10.1038/s41598-019-46979-y}.
    В левом верхнем углу указаны пониженные температуры.}
  \label{MACR-Figure4}
\end{figure}

\section{Результаты}
\label{MACR-SecResults}

Результаты расчета границ сосуществования газа и жидкости показаны на рис.~\ref{MACR-Figure3}(а).
Цветные точки обозначают бинодали, треугольники -- срединные точки.
Точки, которые использовались для аппроксимации [с использованием уравнения~(\ref{MACR-eq4})], выделены сплошной серой линией.
Экстраполированные бинодали обозначены пунктирной серой линией.
Для каждой рассматриваемой системы температура и плотность выражаются в единицах температуры и плотности тройной точки соответственно.
Последние значения вместе с параметрами критических точек приведены в Таб.~\ref{MACR-Table2}.

\begin{table}[h!]
  \centering{
    \begin{tabular}{C{1.5cm}|C{1.0cm}|C{1.0cm}|C{1.0cm}|C{1.0cm}|C{1.0cm}|C{1.0cm}}
      LJn-m & $T_{\rm CP}$ & $\rho_{\rm CP}$ & $T_{\rm TP}$ & $\rho_{\rm TP}$ & $A$ & $a$ \\ \hline
      LJ12-4 & 4.85 & 0.291 & 1.75 & 0.952 & 0.559 & 0.107 \\
      LJ12-5 & 2.18 & 0.304 & 1.03 & 0.867 & 0.804 & 0.208 \\
      LJ12-6 & 1.29 & 0.315 & 0.72 & 0.830 & 1.002 & 0.326 \\
      LJ16-6 & 1.55 & 0.316 & 0.98 & 0.816 & 0.969 & 0.334 \\
      Ethane & 305.3 & 206.7 & 90.34 & 651.9 & 113.1 & 1.158
    \end{tabular}

  }
  \caption{Значения плотностей и температур критических и тройных точек и параметры аппроксимации по уравнению~\eqref{MACR-eq4} для рассматриваемых моделей.
    Для обобщенных систем LJ температуры и плотности даны в сокращенных единицах.
    Для этана температура выражена в К, а плотность выражена в $\text{кг}/\text{м}^3$.
    Параметры критической и тройной точек для этана взяты из работы~\cite{10.1063/1.555785}.}
  \label{MACR-Table2}
\end{table}

\begin{figure}[!t]
  \centering
  \includegraphics[width=160mm]{MACR-Figure5.pdf}
  \caption{Результаты для потенциала LJ$12$-$4$.
    Рисунок аналогичен рисунку~\ref{MACR-Figure4}(a)-(f).}
  \label{MACR-Figure5}
\end{figure}


Замечено, что с увеличением дальнодействия потенциала увеличиваются как температуры тройных и критических точек, так и их отношение $T_{\rm CP}/T_{\rm TP}$.

Затем по рассчитанным фазовым диаграммам была вычислена подвижность частиц при плотностях и температурах, соответствующих бинодали жидкости.
Полученная зависимость подвижности частиц от температуры представлена на рис.~\ref{MACR-Figure3}(b).
Цветные точки на (b) соответствуют цветным точкам на (a).
Серые точки обозначают подвижности на экстраполированных частях бинодали.

\begin{figure}[!t]
  \centering
  \includegraphics[width=160mm]{MACR-Figure6.pdf}
  \caption{Результаты для потенциала LJ$12$-$5$.
    Рисунок аналогичен рисунку~\ref{MACR-Figure4}(a)-(f).}
  \label{MACR-Figure6}
\end{figure}

Отметим, что при низких температурах подвижность на бинодали имеет линейную зависимость от температуры.
Ее наклон увеличивается с уменьшением дальнодействующего характера потенциала взаимодействия (т.е. с увеличением показателя притяжения).
Линейная зависимость сохраняется до определенной температуры, после которой происходит отклонение от линейной зависимости.
Возникновение такой нелинейности может быть связано с особенностями коллективной динамики частиц, которые должны коррелировать со спектрами коллективных возбуждений.

Вычисленные спектры системы LJ$12$-$6$ показаны на рис.~\ref{MACR-Figure4}.
На рисунке~\ref{MACR-Figure4}(а) изображены зависимости подвижности от температуры, а черными стрелками указаны температуры, при которых рассчитывались спектры. 
Были выбраны точки вблизи температуры, при которой наблюдается начало нелинейной зависимости, а также температура вблизи тройной точки.
На рисунке~\ref{MACR-Figure4}(b)-(f) показаны вычисленные дисперсии продольных и поперечных мод в этих точках.
Красный цвет соответствует модели с двумя осцилляторами, а серый — одномодовому анализу~\cite{10.1038/s41598-019-46979-y}.

Нетрудно заметить, что по мере приближения температуры к точке, соответствующей возникновению нелинейной зависимости, дисперсионные соотношения демонстрируют переход от осциллирующей к монотонной зависимости от волнового числа.
Таким образом, качественное изменение температурной зависимости подвижности частиц сопровождается изменением спектров возбуждения.
Наблюдаемая картина не является особенной для системы LJ$12$-$6$.
Аналогичная тенденция замечена и в других исследованных обобщенных ЛД-систем.
Это наблюдение дает новое свидетельство тесной связи между диффузией и свойствами коллективных возбуждений.


\begin{figure}[!t]
  \centering
  \includegraphics[width=160mm]{MACR-Figure7.pdf}
  \caption{Результаты для потенциала LJ$16$-$6$.
    Рисунок аналогичен рисунку~\ref{MACR-Figure4}(a)-(f).}
  \label{MACR-Figure7}
\end{figure}


\section{Заключение главы}
\label{MACR-SecConclusions}

В данной главе исследовано влияние формы потенциала парного взаимодействия на фазовые диаграммы и подвижность частиц в жидкой фазе.
Были рассчитаны кривые сосуществования газа и жидкости для потенциалов с переменным дальнодействием притяжения.
Отмечено, что с увеличением дальнодействующего характера потенциала температуры тройной и критической точек, а также их отношение $T_{\rm CP}/T_{\rm TP}$ увеличиваются.
Коэффициент диффузии и обратный ему коэффициент подвижности вычислялись на жидких бинодалиях.
Было обнаружено, что температурная зависимость подвижности линейна в широком диапазоне температур с тем большим наклоном, чем меньше диапазон притяжения.
Кроме того, установлено, что начало нелинейной температурной зависимости подвижности при высоких температурах совпадает с переходом дисперсионных зависимостей коллективных возбуждений от осциллирующей к монотонной зависимости от волнового числа.
Полученные результаты дают возможность для дальнейшего изучения диффузии и ее связи с коллективными процессами в конденсированных многочастичных системах.

\newpage
\newpage
\begin{center}
\textbf{\large ЗАКЛЮЧЕНИЕ}
\end{center}
\refstepcounter{chapter}


\addcontentsline{toc}{chapter}{ЗАКЛЮЧЕНИЕ}
Основные результаты магистерской квалификационной работы:
\begin{enumerate}

    \item Проведен литературный обзор по теме "влияния дальнодействия притяжения на различные физические параметры".

    \item Был продемонстрирован новый подход к описанию плавления в молекулярных системах, основанный на $\lambda^2$--параметре, рассчитываемый на основе разбиения системы частиц на ячейки вороного. Показано, что разработанная модель демонстрирует существенно нелинейное поведение. Кроме того, предложенная модель позволяет с высокой степенью детализации изучать зародышеобразование в различных режимах перегрева и эволюцию жидких зародышей.

    \item Исследовано влияние формы потенциала парного взаимодействия на фазовые диаграммы и подвижность частиц в жидкой фазе. Рассчитаны кривые сосуществования газа и жидкости для потенциалов с переменной силой притяжения. Установлено, что с увеличением дальнодействия потенциала температуры тройной и критической точек, а также их отношение $T_{\rm CP}/T_{\rm TP}$ также увеличиваются. Были расчитаны коэффициент диффузии и коэффициент подвижности на жидких бинодалиях. Обнаружено, что температурная зависимость подвижности линейна в широком диапазоне температур с тем большим наклоном, чем меньше диапазон притяжения. Установлено, что начало нелинейной температурной зависимости подвижности при высоких температурах совпадает с переходом дисперсионных зависимостей коллективных возбуждений от осциллирующего к монотонному виду.

    \item Разработан новый метод распознавания фаз, основанный на алготиме кластеризации DBSCAN. В совокупности с алгоритмом выделения поверхности метод позволяет с высокой точность расчитывать фазовые диаграммы систем с различной плотностью и формами кластеров. Проведен сравнительный анализ различных методов построения фазовых диаграмм. Показано, что новый метод распознавания фаз может быть применен к переохлажденным системам частиц для анализа скорости нуклеации при различном дальнодействии притяжения.

\end{enumerate}

\newpage

%\bibliographystyle{unsrt}
\bibliographystyle{biblio/gost2008n}
\renewcommand\bibname{\normalsize СПИСОК ИСПОЛЬЗОВАННЫХ ИСТОЧНИКОВ}
\let\BibEmph=\emph
%\bibliographystyle{ugost2008}  %% стилевой файл для оформления по ГОСТу
\bibliography{biblio/biblio}

\end{document}
