% ===================================================================
% ОСНОВНЫЕ НАСТРОЙКИ ДОКУМЕНТА
% ===================================================================
\documentclass[
    candidate, % тип документа (вероятно, кандидатская диссертация)
    subf,      % использование пакета subfig для вложенных рисунков
    times      % использовать Times в качестве основного шрифта
]{disser}

% ===================================================================
% ЯЗЫК, КОДИРОВКА И ШРИФТЫ
% ===================================================================

% --- Язык и кодировка ---
\usepackage[T2A]{fontenc}   % Поддержка кириллических шрифтов в LaTeX
\usepackage[utf8]{inputenc} % Кодировка исходного .tex файла (стандарт для современных систем)
\usepackage[english,russian]{babel} % Основной язык - русский, с возможностью переключения на английский

% --- Настройка шрифтов ---
% Установка основного шрифта Times New Roman.
\renewcommand{\rmdefault}{ftm}

% ===================================================================
% ГЕОМЕТРИЯ СТРАНИЦЫ И МАКЕТ
% ===================================================================

% --- Поля страницы ---
\usepackage[left=3cm, right=1cm, top=2cm, bottom=2cm]{geometry}

% --- Междустрочный интервал ---
\usepackage[onehalfspacing]{setspace} % Полуторный интервал

% --- Колонтитулы ---
\usepackage{fancyhdr} % Мощный инструмент для настройки колонтитулов
\pagestyle{fancy}     % Применяем стиль fancy ко всему документу
\fancyhf{}            % Очищаем все поля верхних и нижних колонтитулов
\fancyfoot[C]{\thepage} % Номер страницы по центру в нижнем колонтитуле
\renewcommand{\headrulewidth}{0pt} % Убираем линию в верхнем колонтитуле

% ===================================================================
% РАБОТА С ГРАФИКОЙ
% ===================================================================
\usepackage{graphicx}       % Для вставки изображений (\includegraphics)
\graphicspath{{images/}}    % Указываем папку, где LaTeX будет искать изображения
\usepackage{pdfpages}       % Для вставки страниц из других PDF-файлов
\usepackage{epstopdf}       % Автоматически конвертирует EPS в PDF при компиляции

% ===================================================================
% МАТЕМАТИЧЕСКИЕ ПАКЕТЫ
% ===================================================================
\usepackage{amsmath}   % Расширенные возможности для формул (например, align, gather)
\usepackage{amssymb}   % Дополнительные математические символы
\usepackage{bm}        % Корректное полужирное начертание для математических символов (\bm)

% ===================================================================
% ТАБЛИЦЫ
% ===================================================================
\usepackage{array}     % Расширенные возможности для столбцов
\usepackage{tabularx}  % Таблицы с автоматическим переносом текста в ячейках
\usepackage{dcolumn}   % Выравнивание чисел в столбцах по десятичному разделителю
\usepackage{multirow}  % Объединение ячеек в нескольких строках

% ===================================================================
% БИБЛИОГРАФИЯ И ГИПЕРССЫЛКИ
% ===================================================================
% Важно: hyperref рекомендуется загружать одним из последних,
% так как он переопределяет множество команд.
\usepackage{cite} % Улучшает отображение цитат, группируя их (например, [1-3])

% ===================================================================
% ВСПОМОГАТЕЛЬНЫЕ ПАКЕТЫ
% ===================================================================
\usepackage{listings}  % Для красивой вставки листингов кода
\usepackage{afterpage} % Позволяет размещать материал (например, рисунок) после текущей страницы

% ===================================================================
% ПОЛЬЗОВАТЕЛЬСКИЕ КОМАНДЫ И НАСТРОЙКИ
% ===================================================================

% --- Настройка подписей к рисункам и таблицам ---
\usepackage[font={normal}]{caption} % Управление стилем подписей
\captionsetup{
    format=hang,      % "Висячая" строка для многострочных подписей
    labelsep=period   % Точка после номера (e.g., "Рис. 1.")
}

% --- Пользовательские типы столбцов для таблиц ---
% Создает новый тип столбца 'C', который центрирует содержимое по горизонтали.
% Использование: C{ширина}, например, C{3cm}
\newcommand{\PreserveBackslash}[1]{\let\temp=\\#1\let\\=\temp}
\newcolumntype{C}[1]{>{\PreserveBackslash\centering}p{#1}}

% --- Пользовательские математические команды ---
% Определение команды для векторов. \bm из пакета bm работает корректнее,
% чем \mathbf, так как сохраняет курсивное начертание для букв.
% Использование: \vect{v}
\newcommand{\vect}[1]{\bm{#1}}

% --- Глубина оглавления ---
% Включаем в оглавление разделы и подразделы (но не subsubsection)
\setcounter{tocdepth}{2}

% ===================================================================
% HYPERREF
% ===================================================================
\usepackage{hyperref}
